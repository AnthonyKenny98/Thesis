% @Author: AnthonyKenny98
% @Date:   2020-03-01 12:32:47
% @Last Modified by:   AnthonyKenny98
% @Last Modified time: 2020-03-01 13:02:48
\begin{table}[H]
\begin{center}
\begin{tabular}{|p{.2\linewidth}|p{.2\linewidth}|p{.6\linewidth}|}
    \hline
    \textbf{Component}          & \textbf{Source}   & \textbf{Description} \\
    \hline
    \multicolumn{3}{|c|}{RISC-V Instruction Set} \\
    \hline
    \ac{RV32I}              & Berkeley & 40 Instructions defined such that \ac{RV32I} is sufficient to form a compiler target and suport modern operating systems \cite{Waterman2019}. \\
    \hline
    Extension        & \textit{New} & This is the custom extension defined by this thesis targeting motion planning instructions. It is outlined in Chapter \ref{chap:RiscvProcessor}. \\
    \hline
    \multicolumn{3}{|c|}{C-Implementation of RRT} \\
    \hline
    RRT        & \textit{New} & Due to lack of available implementations of \ac{RRT} suitable for the purposes of this thesis, \ac{RRT} was implemented from the ground up in C. This is detailed in Chapter \ref{chap:MotionPlanningInSoftware} \\
    \hline
    \multicolumn{3}{|c|}{FPGA Synthesized Chip} \\
    \hline
    Zynq-7000        & Xilinx & The Zynq-7000 family of \ac{SoC}s are a low cost FPGA and \ac{ARM} combined unit. \\
    \hline
    PhilosophyV     & \textit{New} & The processor built for this thesis to demonstrate how the RISC-V extension and hardware unit work together. This is detailed in Chapter \ref{chap:RiscvProcessor} \\
    \hline
    HoneyBee        & \textit{New} & The functional unit designed specifically for faster execution of edge collision detection computations. Outlined in Chapter \ref{chap:MotionPlanningInHardware} \\
    \hline
\end{tabular}
\caption{List of System Components and their Descriptions}
\label{table:componentList}
\end{center}
\end{table}