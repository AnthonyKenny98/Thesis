% @Author: AnthonyKenny98
% @Date:   2020-03-01 12:32:47
% @Last Modified by:   AnthonyKenny98
% @Last Modified time: 2020-04-11 00:49:38
\begin{table}[t!]
\begin{center}
\begin{tabular}{|p{.2\linewidth}|p{.2\linewidth}|p{.5\linewidth}|}
    \hline
    \textbf{Component}          & \textbf{Source}   & \textbf{Description} \\
    \hline
    \multicolumn{3}{|c|}{C-Implementation of RRT} \\
    \hline
    Rapidly-exploring Random Tree        & \textit{A.J.W. Kenny} & Due to lack of available implementations of \gls{RRT} suitable for the purposes of this thesis, \gls{RRT} was implemented from the ground up in C. This is detailed in Chapter \ref{chap:MotionPlanningInSoftware} \\
    \hline
    \multicolumn{3}{|c|}{RISC-V Instruction Set} \\
    \hline
    \gls{RV32I}              & Berkeley & 40 Instructions defined such that \gls{RV32I} is sufficient to support modern operating systems \cite{Waterman2019}. \\
    \hline
    Motion Planning Extension        & \textit{A.J.W. Kenny} & This is the custom extension defined by this thesis targeting motion planning instructions. It is outlined in Chapter \ref{chap:MotionPlaningArchitecture}. \\
    \hline
    \multicolumn{3}{|c|}{Motion Planning Specific Processor} \\
    \hline
    PhilosophyV     & \textit{A.J.W. Kenny} & The processor built for this thesis to demonstrate how the RISC-V extension and hardware unit work together. This is detailed in Chapter \ref{chap:MotionPlaningArchitecture} \\
    \hline
    HoneyBee        & \textit{A.J.W. Kenny} & The functional unit designed specifically for faster execution of edge collision detection computations. Outlined in Chapter \ref{chap:MotionPlanningInHardware} \\
    \hline
\end{tabular}
\mycaption{List of System Components and their Descriptions}{}
\label{table:componentList}
\end{center}
\end{table}