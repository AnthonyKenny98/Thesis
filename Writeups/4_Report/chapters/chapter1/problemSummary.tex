% @Author: AnthonyKenny98
% @Date:   2020-02-23 12:45:54
% @Last Modified by:   AnthonyKenny98
% @Last Modified time: 2020-03-01 00:13:18

\subsection{Background}
    
    The \ac{UAV} has been utilised in military applications extensively throughout the late 20th and early 21st century. However, over the last decade, their use in non-military uses, such as commerical, scientific, agricultural, and recreational, such that the number of civilian drones vastly outnumber military \ac{UAV}s.\todo{cite} Particularly in the commercial sector, such rapid growth in the number and range of applications means that autonomy is key for the profitable adoption of \ac{UAV}s. Such autonomy relies on efficient computation of motion planning algorithms. However, the implementation of these algorithms can be quite computationally expensive, and thus slow and/or detrimentally power consuming. As such, this thesis aims to design specialized hardware to more efficiently compute motion plans for autonomous drones.

    \subsubsection*{Robotics}
        For well over 2000 years, the concept of robotics, albeit not always with such a term, has fascinated humans. As early as the first century A.D., the Greek mathematician and engineer, Heron of Alexandria, described more than 100 different machines and automata in \textit{Pneumatica} and \textit{Automata}\cite{Alexandrinus}. In 1898, Nikola Tesla demonstrated the first radio-controlled vessel. Since then, the world has seen widespread application of robotics in manufacturing, mining, transport, exploration, and weaponry. For the last few decades, robots have operated in controlled, largely unchanging environments (e.g.\ an assembly line) where their environment and movements are largely known \textit{a priori}.
        \newline
        However, in recent years a new generation of autonomous robots has been developed for a wide range of real-world, complex applications. The increasing trend the use of autonomous robots is shown in Figure \ref{fig:useOfAutonomousRobots}. These new robots, unlike those traditional ones described above, are required to adapt to the changing environment in which they operate. As such, they must perform motion planning in real time.

        % @Author: AnthonyKenny98
% @Date:   2020-02-23 12:12:56
% @Last Modified by:   AnthonyKenny98
% @Last Modified time: 2020-03-01 13:01:50

\begin{figure}%[H]
\begin{center}
% \missingfigure[figwidth=\linewidth]{Some sort of line/bar graph showing the increasing use of Autonomous robots over time. Need to find}
\includegraphics[width=0.8\linewidth]{img/sampleLineGraph.png}
\caption{The use of Autonomous Robots over time [TODO]}
\label{fig:useOfAutonomousRobots}
\end{center}
\end{figure}

    \subsubsection*{Motion Planning}
        \todo[inline]{TODO: More of an introduction to motion planning.}
        
        Motion Planning refers to the problem of determining how a robot moves through a space to acheive a goal. Chapter \ref{chap:MotionPlanningInSoftware} provides a detailed explanation of motion planning and of \ac{RRT}, a commonly used motion planning algorithm.
        \newline
        On the algorithmic level, motion planning has been extensively studied and many solutions exist. However, current algorithms running on regular \ac{CPU}s are too slow to execute in real time for robots operating in complex environments. Simply solving this problem with more raw computing power, using energy hungry \ac{GPU}s may have merit in tethered robots. On the other hand, untethered applications, such as autonomous drones, where limiting power consumption is a primary concern, this strategy is infeasible.
        
    \subsection*{Hardware Acceleration}
        Hardware acceleration refers to the strategy of using computer hardware specifically designed to execute a function more efficiently than can be achieved by software running on a general purpose \ac{CPU}.
        Specialized hardware designed to perform specific functions can yield significantly higher performance than software running on general purpose processors, and lower power consumption than \ac{GPU}s. \\
        To take a step back, consider a typical computer implementation hierarchy, demonstrated in Figure \ref{fig:computerHierarchy}. \textbf{User level applications}, such as Google Chrome, Microsoft Word, and Apple's iTunes, sit at the top of the abstraction hierarchy. These applications are implemented in \textbf{High-Mid Level Languages}, such as C/C++, Python, Java, etc. These programming languages have their own hierarchy, but for the purpose of this thesis, it is sufficient to understand that these programming languages are then compiled into \textbf{Assembly Language}. Assembly language closely follows the execution of instructions on the \textbf{processor}, and is defined by an \textbf{\ac{ISA}}. An \ac{ISA} can be thought of as the contract between software programmers and processor engineers, agreeing what instructions the processor is able to implement. This assembly code is finally loaded into the processor's instruction memory and executed. 
        % @Author: AnthonyKenny98
% @Date:   2020-02-29 23:52:30
% @Last Modified by:   AnthonyKenny98
% @Last Modified time: 2020-04-10 12:37:43
\begin{figure}[H]
\begin{center}
\includegraphics[width=\linewidth]{chapters/chapter1/img/computerHierarchy.png}
\mycaption{Simple Visualization of Computer Implementation Hierarchy}{}
\label{fig:computerHierarchy}
\end{center}
\end{figure}

    \subsubsection{RISC-V}
        \todo[inline]{TODO: Introduction to RISC-V and its merits in this problem}

    \todo[inline]{Move the Hardware and RISC-V subsubsections to proposed solution?}   


\subsection{Problem Definition}

    \subsubsection*{Problem Statement}
    \todo[inline]{Revise problem statement}
    Current processors cannot compute motion planning algorithms quickly enough for robots to operate in high complexity environments. Autonomous drones are a specific case of robots requiring real-time motion planning in complex environments. The state-of-the-art strategy of using a Graphics Processing Unit (GPU) to accelerate the execution of these algorithms requires too much power to be cost-effective or feasible for drones to sustain flight for useful periods of time.

    \subsubsection*{End User}
    The end user of this project is a developer of autonomous drones. Such developers have a need for computing hardware that executes motion planning algorithms faster and more power efficiently than existing methods. This thesis will provide a processor design that is synthesizable on an \ac{FPGA}, giving developers a processer for which a \ac{RTOS}, or bare metal code, can be written. 
    Additionally, these developers have a requirement that using a new processor for a drone will not require a massive investment in re-development. As such, this thesis will provide the toolchain necessary to compile C code into executable instructions on the new processor.
    \todo[inline]{TODO: Revise End User}