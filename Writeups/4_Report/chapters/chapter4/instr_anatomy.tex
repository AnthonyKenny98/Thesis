% @Author: AnthonyKenny98
% @Date:   2020-04-10 19:44:03
% @Last Modified by:   AnthonyKenny98
% @Last Modified time: 2020-04-10 20:15:03
Every instruction, is represented in ``assembly language'' as something like ``\texttt{addi rd, rs, 10}''. This is a (somewhat) human readable representation of the instruction. \texttt{addi} computes \texttt{rs + 10} and stores the result in \texttt{rd}. However, this instruction is represented in the processor as a 32-bit sequence of binary values. Imagine an empty 32-bit instruction, such as in Figure \ref{fig:instr_blank}.

        \begin{figure}[H]
        \begin{center}
        \resizebox{\textwidth}{!}{
        \begin{tabular}{c}
            \begin{small}
            \begin{tabular}{llllllllllllllllllllllllllllllll}
                31&&&&&&&& 23&&&&&&&& 15&&&&&&&&7 &&&&&&&0 \\
            \end{tabular}
            \end{small} \\
        
            \begin{tabular}{|l|l|l|l|l|l|l|l|l|l|l|l|l|l|l|l|l|l|l|l|l|l|l|l|l|l|l|l|l|l|l|l|}
                \hline
                &&&&&&&& &&&&&&&& &&&&&&&& &&&&&&& \\
                \hline
            \end{tabular} \\
            
            \begin{small}
            \begin{tabular}{llllllllllllllllllllllllllllllll}
                &&&&&&&& &&&&&&&& &&&&&&&& &&&&&&& \\
            \end{tabular}
            \end{small} \\

        \end{tabular}}
        \mycaption{Empty 32-Bit Instruction}{. Reads \gls{MSB} to \gls{LSB}}
        \label{fig:instr_blank}
        \end{center}
        \end{figure}

The first piece of information to put in the instruction is the \textbf{opcode}. Opcode (short for operation code) is the instructions identity. It tells the processor which operation to execute. Continuing with the \texttt{add} instruction as an example, its opcode is the 7-bit sequence \texttt{0010011}. In RV32I, the opcode is stored in the 7 \glspl{LSB} of the instruction, as shown in Figure \ref{fig:instr_op}

    % \begin{figure}[H]
    % \begin{center}
    % \includegraphics[options]{name}
    % \mycaption{32-Bit Instruction Containing Opcode}{}
    % \label{fig:instr_op}
    % \end{center}
    % \end{figure}
        
The next information to add to the instruction are the \textbf{source and destination registers}. RV32I registers are addressed by 5-bit values. For the \texttt{addi} instruction, they formatted in the instruction as shown in Figure \ref{fig:instr_rs}

    % \begin{figure}[H]
    % \begin{center}
    % \includegraphics[options]{name}
    % \mycaption{32-Bit Instruction Containing Opcode}{}
    % \label{fig:instr_op}
    % \end{center}
    % \end{figure}

Finally, the last piece of information to include is what is known as the \textbf{immediate}. An immediate is just a number \- in this example it is the number 10. In RV32I, numbers are represented with 32 bits, but there aren't 32 bits left in the instruction. As such, the \glspl{MSB} of the immediate are cut off and the \glspl{LSB} added to the instruction.