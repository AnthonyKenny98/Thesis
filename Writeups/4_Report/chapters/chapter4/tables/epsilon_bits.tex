% @Author: AnthonyKenny98
% @Date:   2020-04-09 19:26:20
% @Last Modified by:   AnthonyKenny98
% @Last Modified time: 2020-04-09 20:07:12
\begin{table}[H]
\begin{center}
\begin{tabular}{|c|c|c|}
\hline
\textbf{$\epsilon$} & \textbf{Bits} & \textbf{Registers} \\
\hline
1 & 1 & 1/32 bits\\
\hline
2 & 8 & 8/32 bits \\
\hline
4 & 64 & 2 \\
\hline
6 & 216 & 6.75 \\
\hline
8 & 512 & 16 \\
\hline
\end{tabular}
\mycaption{Required Bits to Represent Output Collisions For Different Values of Epsilon}{. $\epsilon$ values of 1 and 2 underutilise both the register space available and the benefits of more parallelization that would come from larger values. Values larger than 4 would use up far too many of the available registers. $\epsilon = 4$ completely utilizes only two registers.}
\label{table:epsilon_bits}
\end{center}
\end{table}