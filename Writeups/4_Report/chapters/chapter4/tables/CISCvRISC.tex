% @Author: AnthonyKenny98
% @Date:   2020-04-09 11:26:10
% @Last Modified by:   AnthonyKenny98
% @Last Modified time: 2020-04-09 18:23:43
\begin{table}[H]
\begin{centering}
\begin{tabular}{|m{0.45\linewidth}|m{0.45\linewidth}|}
\hline
\textbf{CISC} & \textbf{RISC} \\
\hline
Emphasis on Hardware Implementation 
    & Emphasis on Software \\
\hline
Multi-cycle, complex instructions. Different Instructions take different amounts of time to execute.
    & Single-cycle, simple instructions. All base instructions take the same amount of time to execute. \\
\hline
Operations can be performed directly on values stored in memory. 
    & Memory must be loaded into registers, operated on, and then stored back into memory. \\
\hline
Higher number of cycles per second 
    & Lower number of cycles per second \\
\hline
Smaller Assembly code sizes
    & Larger code sizes \\
\hline
\end{tabular}
\mycaption{Comparison of CISC and RISC ISAs.}{The operating philosophy of the two can really be broken down as follows: CISC has more complex instructions, higher cycles per second, and more cycles per instruction. RISC has fewer, more simple instructions, fewer cycles per second, and generally only one execution cycle per instruction.}
\end{centering}
\end{table}