% @Author: AnthonyKenny98
% @Date:   2020-03-01 14:11:33
% @Last Modified by:   AnthonyKenny98
% @Last Modified time: 2020-03-01 14:33:00
\begin{table}[H]
\begin{center}
\begin{tabular}{|p{.3\linewidth}|p{.64\linewidth}|}
    \hline
    \textbf{Element}             & \textbf{Description/Justification} \\
    \hline
    \multicolumn{2}{|c|}{Contstraints} \\
    \hline
    Dimension $N$  & $N$ defines the dimension of the cubic configuration space for which the functional unit should take an identically sized \ac{OGM}\\
    \hline
    \multicolumn{2}{|c|}{Inputs} \\
    \hline
    Edge $e$  & An Edge $e$ defined for a \ac{3D} configuration space by two points $\{p1, p2\}$, each defined by a set of \ac{3D} coordinates $\{x,y,z\}$.\\
    \hline
    Space $S$ & In an abstract sense, the edge collision detection function takes Space $S$ as an input. In a more practical sense, the functional unit will take an $N\times N\times N$ Occupancy Grid Map \\
    \hline
    Control Inputs & The functional unit must also have ports for control signals such as clock, reset, start, etc. These are required for adding the unit to a processor. \\
    \multicolumn{2}{|c|}{Outputs} \\ 
    \hline
    Return Value & 1 bit return value: 1 if collision, 0 otherwise.\\
    \hline
    Control Outputs & Output ports for control signals such as idle, done, ready, etc. These are required for adding the unit to a processor. \\
    \hline
\end{tabular}
\caption{Technical Specifications for Edge Collision Detection Unit}
\label{table:HoneyBeeTechSpecs}
\end{center}
\end{table}
\todo[inline, caption={Improve Technical Specifications}]{More detail on control units.}