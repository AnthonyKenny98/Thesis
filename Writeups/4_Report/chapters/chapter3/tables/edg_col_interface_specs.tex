% @Author: AnthonyKenny98
% @Date:   2020-03-01 14:11:33
% @Last Modified by:   AnthonyKenny98
% @Last Modified time: 2020-04-07 14:09:51
\begin{table}[H]
\begin{center}
\begin{tabular}{|p{.2\linewidth}|p{.74\linewidth}|}
    \hline
    \textbf{Element}             & \textbf{Description/Justification} \\
    \hline
    \multicolumn{2}{|c|}{Constraints} \\
    \hline
    Length $\epsilon$  & $\epsilon$ defines the max edge length. The space being checked and the output sequence has the dimensions $\epsilon\times\epsilon\times\epsilon$ \\
    \hline
    \multicolumn{2}{|c|}{Inputs} \\
    \hline
    Edge $e$  & An Edge $e$ defined for a \gls{3D} \gls{configuration} space by two points $\{p1, p2\}$, each defined by a set of \gls{3D} coordinates $\{x,y,z\}$.\\
    \hline
    Control Inputs & The functional unit must have ports for control signals: clock, reset, start. These are required for adding the unit to a processor. \\
    \hline
    \multicolumn{2}{|c|}{Outputs} \\ 
    \hline
    Return Value & $\epsilon^3$ bit sequence: 1 if collides with grid at that index, 0 otherwise.\\
    \hline
    Control Outputs & Output ports for control signals: idle, done, ready. These are required for adding the unit to a processor. \\
    \hline
\end{tabular}
\mycaption{Interface Specifications for Edge Collision Detection Unit}{}
\label{table:edg_col_interface_specs}
\end{center}
\end{table}