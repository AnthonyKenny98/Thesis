% @Author: AnthonyKenny98
% @Date:   2020-02-28 15:02:19
% @Last Modified by:   AnthonyKenny98
% @Last Modified time: 2020-02-28 23:28:59

\todo[inline]{Brief introduction outlining purpose of performance analysis}

\subsection{Methodology}
    To restate, the aim of this thesis is to design a computer processor with reduced execution time of motion planning algorithms, such as \ac{RRT}. As such, it is important to understand the elements of the algorithm that have the highest percentage of CPU execution time. To determine this, it was necessary to implement my own, naive but typical, \ac{RRT} in C. This program could then be compiled and analysed using a software performance profiling tool. With this, I could design experiments to determine the critical RRT functions (those occupying a majority of CPU time) and see how this varies given different parameters.
    \todo[inline]{Outline of method of analysis. Something better than the above}

    \subsubsection{VTune Profiler}
    \label{subsubsection:vtune}
        VTune Profiler performance profiler is an application for software performance analysis. It provides functionality to examine hot-spots for CPU execution time through a top down analysis, shown below in Figure \ref{figure:VTuneTopDown}. As can be seen from the figure, the top down analysis tool shows the percentage of CPU time taken up by each function. I used this tool to profile the algorithm's performance as I changed certain parameters.
        \todo[inline]{Rewrite the above}
        % @Author: AnthonyKenny98
% @Date:   2020-02-23 14:33:19
% @Last Modified by:   AnthonyKenny98
% @Last Modified time: 2020-02-23 14:36:06

\begin{figure}[H]
\begin{center}
    \missingfigure[figwidth=\linewidth]{Screenshot of VTune Top Down Analysis (Maybe)}
    \caption{VTune Amplifier TopDown Analysis Example}
    \label{figure:VTuneTopDown}
\end{center}
\end{figure}

    \subsubsection{Internal Timing}
        The limitation of VTune Profiler is that it can only profile software running on Intel processors, which implement the x86-64 \ac{ISA}. As such, when the time comes to analyse performance of the software running on a RISC-V processor, another method will be required. A simple and effective way of measuring execution performance is to insert timing functionality into the software itself. 

    \subsubsection{Comparison}

\subsection{Results}
\label{section:rrt_analysis_results}