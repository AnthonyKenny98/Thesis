% @Author: AnthonyKenny98
% @Date:   2020-02-22 15:53:59
% @Last Modified by:   AnthonyKenny98
% @Last Modified time: 2020-03-01 13:08:41

% INTRO
\ac{RRT} is an algorithm designed to efficiently search, and thus plan a path through, a high-complexity environment by randomly sampling points and building a tree. The algorithm randomly samples points, draws an edge from the nearest currently existing node in the tree, to grow the tree in the space. It is inherently biased to grow towards large unsearched areas of the problem. RRT was developed by S. LaVelle\cite{LaValle1998} and J. Kuffner\cite{LaValle2001}. It is used in autonomous robotic motion planning problems such as autonomous drones, the focus of this thesis.

% ALGORITHM
\subsection{Algorithm}
    
    % BASIC RRT - BUILDING THE TREE
    \subsubsection*{Building the Tree}

        Put simply, \ac{RRT} builds a tree (referred to as a graph) of possible configurations, connected by edges, for a robot of some physical description. It does so by randomly sampling the configuration space and adding configurations to the graph. From this graph, a path from the initial configuration to some goal configuration can be found, given a high enough number of iterations. As such, \ac{RRT} can be considered probabalistically complete.
        The pseudo-code for \ac{RRT} can be seen in Algorithm \ref{algorithm:rrt}
        
        % RRT Algorithm
        % @Author: AnthonyKenny98
% @Date:   2020-02-22 16:15:38
% @Last Modified by:   AnthonyKenny98
% @Last Modified time: 2020-02-23 14:06:05

\begin{algorithm}[H]
    \caption{Rapidly-exploring Random Tree with Collision Detection}
    \SetAlgoLined
    \SetArgSty{textnormal}
    \textbf{Inputs:} Space $S$ with obstacles \\ 
    \textbf{Output:} Collision free graph $G$ with $K$ nodes \& edges \\
        $G$.init()\;
        \For{$k = 1$ to $K$}{
            \While { !\text{pointCollision}($node_{new}$) } {
                $q_{rand} \leftarrow $ getRandomNode(); \\
                $q_{near} \leftarrow $ findNearestNode(); \\
                $q_{new} \leftarrow $ stepFromTo(); \\
            }
            $e_{new} \leftarrow $ newEdge($q_{near}, q_{new}$) \\
            \eIf{\text{!edgeCollision($e_{new}$)}} {
                $G$.addNode($q_{new}$); \\
                $G$.addEdge($e_{new}$);
            }{
                $k = k-1$;
            }
        }
\label{algorithm:rrt}
\end{algorithm}

\todo[inline]{Relook at above algorithm}

        % Explanation of Algorithm, referencing visual step by step figure
        Algorithm \ref{algorithm:rrt} can be visually represented in Figure \ref{fig:rrt-step-by-step}. Consider a \ac{2D} robot operating in a \ac{2D} workspace. A Graph $G$ is initialized containing an initial configuration, $q_{init}$, with constraints on the number of nodes that the graph can hold, $K$, and the maximum distance between two nodes, $\Delta q$. This is shown in Sub-figure \ref{subfig:rrt-step-by-step-A}. A random configuration for the robot, $q_{rand}$ is generated (\ref{subfig:rrt-step-by-step-B}). The nearest existing configuration in $G$, $q_{near}$, is found. (In the first iteration, $q_{near} = q_{init}$, shown in Sub-figure \ref{subfig:rrt-step-by-step-C}). The distance between $q_{near}$ and $q_{rand}$ is calculated. If this distance is less than $\Delta q$, $q_{new} = q_{rand}$. If not, $q_{new}$ is selected, typically by moving by $\Delta q$ from $q_{near}$ towards $q_{rand}$ (\ref{subfig:rrt-step-by-step-C}). $q_{new}$ is then added to $G$. This is repeated for $K$ configurations.

        % Step By Step RRT Figure
        % @Author: AnthonyKenny98
% @Date:   2020-02-27 14:22:10
% @Last Modified by:   AnthonyKenny98
% @Last Modified time: 2020-02-27 15:36:30

\begin{figure}[H]
\begin{center}
\begin{tabular}{c c}

    % Subfigure A
    \begin{subfigure}{0.45\textwidth}
    \includegraphics[draft=false,width=\linewidth]{chapters/chapter2/img/RRT_step_by_step-A.png}
    \caption{Graph $G$ contains only $q_{init}$ \newline}
    \label{subfig:rrt-step-by-step-A}
    \end{subfigure} &
    % 
    % Subfigure B
    \begin{subfigure}{0.45\textwidth}
    \includegraphics[draft=false,width=\linewidth]{chapters/chapter2/img/RRT_step_by_step-B.png}
    \caption{The first random configuration, $q_{rand}$, is generated}
    \label{subfig:rrt-step-by-step-B}
    \end{subfigure} \\ \\

    % Subfigure C
    \begin{subfigure}{0.45\textwidth}
    \includegraphics[draft=false,width=\linewidth]{chapters/chapter2/img/RRT_step_by_step-C.png}
    \caption{In first iteration, $q_{near} = q_{init}$. Distance between $q_{init}$ and $q_{rand}$ is greater than $\Delta q$, so $q_{new}$ is generated and added to $G$}
    \label{subfig:rrt-step-by-step-C}
    \end{subfigure} &
    % 
    % Subfigure D
    \begin{subfigure}{0.45\textwidth}
    \includegraphics[draft=false,width=\linewidth]{chapters/chapter2/img/RRT_step_by_step-D.png}
    \caption{This is repeated $K$ times. For $G$, $K=10$, and the red line represents the edges between configurations}
    \label{subfig:rrt-step-by-step-D}
    \end{subfigure}

\end{tabular}
    
    % Caption and Label
    \caption{Step by step demonstration of \ac{RRT} Algorithm for 2D robot in 2D space}
    \label{fig:rrt-step-by-step}
\end{center}
\end{figure}

    % COLLISION DETECTION
    \subsubsection{Collision Detection}

        Algorithm \ref{algorithm:rrt} shows how \ac{RRT} builds a graph of possible configurations connected by edges in a free configuration space. However, in real-world applications, a robot's configuration space often contains obstacles. As such, collision detection must be included in the algorithm. The two types of collisions the algorithm must check for are \textit{configuration collisions} (those where the robot would collide with an obstacle in a given configuration) and \textit{edge collisions} (where the robot would collide when moving between two collision free configurations).

        The RRT with configuration and edge collision detection can be seen in Algorithm \ref{algorithm:rrt_collision}. The method of implementing \ac{RRT} with collision detection to model a drone in 3D space is detailed in Section \ref{section:implementation}.

        % @Author: AnthonyKenny98
% @Date:   2020-02-27 18:27:36
% @Last Modified by:   AnthonyKenny98
% @Last Modified time: 2020-02-27 18:34:46
\bigskip
\begin{algorithm}[H]
    \caption{Rapidly-Exploring Random Tree with Collision Detection}
    \SetAlgoLined
    \SetArgSty{textnormal}
    \begin{tabular}{l l}
    \textbf{Inputs:}    & Initial configuration $q_{init}$,\\ 
                        & Number of nodes in graph $K$, \\
                        & Incremental Distance $\Delta q$, \\
                        & Space $S$ containing obstacles \\
    \textbf{Output:}    & RRT Graph $G$ with $K$ configurations $[q]$ \& edges $[e]$ \\
    \end{tabular}

        $G$.init()\;
        \For{$k = 1$ to $K$}{
            \While { !\text{pointCollision}($q_{new}$) } {
                $q_{rand} \leftarrow $ randomConfiguration(); \\
                $q_{near} \leftarrow $ nearestVertex($q_{rand}$, $G$); \\
                $q_{new} \leftarrow $ newVertex($q_{near}$, $q_{rand}$, $\Delta q$); \\
            }
            $e_{new} \leftarrow $ newEdge($q_{near}, q_{new}$) \\
            \eIf{\text{!edgeCollision($e_{new}$)}} {
                $G$.addVertex($q_{new}$); \\  
                $G$.addEdge($q_{near}$, $q_{new}$);
            }{
                $k = k-1$;
            }
        }
\label{algorithm:rrt_collision}
\end{algorithm}
\bigskip

\newpage
\subsection{Implementation}\label{section:implementation}
    
    With \ac{RRT} selected as the benchmark algorithm against which to test specialised hardware, this project required an implementation of the algorithm that satisfied the following criteria.\todo{Better RRT Implementation introductory sentence}

    \begin{table}[H]
\begin{center}
\begin{tabular}{|p{.3\linewidth}|p{.64\linewidth}|}
    \hline
    Requirement             & Description and Justification \\
    \hline
    C/C++ Implementation    & As outlined in Section \ref{subsection:project_structure}, the critical step in determining the design of specialized hardware to accelerate \gls{RRT} is CPU performance analysis of the algorithm to determine computational hot-spots. Implementations in C allow for the use of certain CPU profiling tools, described in Section \ref{subsubsection:vtune}, unlike higher-level languages such as Python. \\
    \hline
    Models Drone as Point   & In reality, implementing \gls{RRT} for a drone would model the robot as a \gls{3D} object defined by coordinates $\{x, y, z\}$ and Euler angles $\{\alpha, \beta, \gamma \}$. However, for simplicity's sake, modelling the drone as a point defined by coordinates $\{x, y, z\}$ will suffice. Time permitting, this could be revisited. \todo[inline]{Change this based on whether time does permit} \\
    \hline
    Mirrors Algorithm       & In order for the results of CPU performance analysis to be easy to understand, software implementation of \gls{RRT} should call functions that mirror the functions described in Algorithms \ref{algorithm:rrt} and \ref{algorithm:rrt_collision}. \\
    \hline
\end{tabular}
\caption{Technical Specifications for \gls{RRT} Implementation}
\label{table:RRT_Tech_Specs}
\end{center}
\end{table}
\todo[inline]{Improve this table}

    The original intention was to find an existing implementation of RRT that could fulfill these requirements. Most open source implementations found online were in Python, and all those implemented in C were unsuitable\cite{RoboJackets2019}\cite{Planning2019}\cite{Sourishg2017}\cite{Vss2sn2019}, as they had extraneous \ac{GUI}s, reliance on external \ac{API}s, and other features that would distort analysis of algorithmic hot-spots.

    As a result, it was necessary to build a C implementation of RRT from the ground up that satisfied the requirements in Table \ref{table:RRT_Tech_Specs}. It can be found in this project's GitHub repository. It follows Algorithm \ref{algorithm:rrt} closely. For monitoring correctness, I build in an optional \ac{GUI} that shows the tree, starting node, and obstacles.

    \subsubsection*{Modelling a \ac{UAV} for RRT}

    \subsubsection{Implementation in 2D}
    The first step was to implement RRT with a 2-Dimensional workspace. \todo[inline]{More detail}
    % @Author: AnthonyKenny98
% @Date:   2020-02-23 14:14:12
% @Last Modified by:   AnthonyKenny98
% @Last Modified time: 2020-03-01 08:06:12


\begin{figure}[H]
\begin{center}
\begin{tabular}{c  c}
    \includegraphics[width=0.45\linewidth]{chapters/chapter2/img/rrt_2d_1.png} & \includegraphics[width=0.45\linewidth]{chapters/chapter2/img/rrt_2d_2.png} \\
    \includegraphics[width=0.45\linewidth]{chapters/chapter2/img/rrt_2d_3.png} & \includegraphics[width=0.45\linewidth]{chapters/chapter2/img/rrt_2d_4.png}
    \end{tabular}
    \caption{2D RRT Implementation shown by \ac{GUI}}
    \label{figure:2DrrtGui}
\end{center}
\end{figure}

    \subsubsection{Implementation in 3D}
    \todo[inline]{Describe implementation in 3D}
    % @Author: AnthonyKenny98
% @Date:   2020-02-23 14:14:12
% @Last Modified by:   AnthonyKenny98
% @Last Modified time: 2020-02-23 14:19:03


\begin{figure}[H]
\begin{center}
    % \begin{tabular}{c  c}
    % \includegraphics[draft=false,width=0.45\linewidth]{chapters/chapter2/img/rrt_2d_1.png} & \includegraphics[draft=false,width=0.45\linewidth]{chapters/chapter2/img/rrt_2d_2.png} \\
    % \includegraphics[draft=false,width=0.45\linewidth]{chapters/chapter2/img/rrt_2d_3.png} & \includegraphics[draft=false,width=0.45\linewidth]{chapters/chapter2/img/rrt_2d_4.png}
    % \end{tabular}
    \missingfigure[figwidth=\linewidth, figheight=10cm]{Simulations of 3D RRT Implementations}
    \caption{3D RRT Implementation shown by \ac{GUI}}
    \label{figure:3DrrtGui}
\end{center}
\end{figure}
