% @Author: AnthonyKenny98
% @Date:   2020-02-22 15:42:12
% @Last Modified by:   AnthonyKenny98
% @Last Modified time: 2020-03-01 14:01:59

This thesis aims to design hardware that executes motion planning algorithms faster than those same algorithms can execute on generic hardware. Chatper 2 introduces motion planning and details the process of implementing and analysing \ac{RRT} to identify computational hot-spots in the algorithm and thus identify the biggest opportunities for hardware optimization. \\
Section \ref{section:motion_planning_background} provides an introduction to Motion Planning Algorithms in general. Section \ref{section:rrt} outlines the \ac{RRT} algorithm, and describes the implementation of \ac{RRT} for this project. Finally, Section \ref{section:rrt_analysis} outlines a method for performance analysis of \ac{RRT} and the results of such analysis.
\todo[inline]{Write a better introduction that more accurately defines sections and makes the POINT of this chapter clear.}

\section{Background} 
\label{section:motion_planning_background}
    Motion Planning Algorithms refer to the set of algorithms that find possible sequences of valid configurations for a robot in a space.
    \newline
    \newline
    \todo[inline,caption={Background on Motion Planning Algorithms.}]{Need more background here. Needs more of a mathematical foundation}

\newpage
\section{Rapidly-Exploring Random Tree}
\label{section:rrt}
    % @Author: AnthonyKenny98
% @Date:   2020-02-22 15:53:59
% @Last Modified by:   AnthonyKenny98
% @Last Modified time: 2020-04-05 18:38:15


% TECHNICAL SPECIFICATIONS
\subsection{Technical Specifications}

    With \gls{RRT} selected as the benchmark algorithm against which to test specialised hardware, this project required an implementation of the algorithm that satisfied the following criteria shown in Table \ref{table:RRT_Tech_Specs_Abbrev}. Appendix \ref{section:rrt_appendix_tech_specs} is a more thorough description of the technical specifications for the implementation of RRT. 

    % Tech Spec Table
    \begin{table}[H]
\begin{center}
\begin{tabular}{|p{.3\linewidth}|p{.64\linewidth}|}
    \hline
    Requirement             & Description and Justification \\
    \hline
    C/C++ Implementation    & As outlined in Section \ref{subsection:project_structure}, the critical step in determining the design of specialized hardware to accelerate \gls{RRT} is CPU performance analysis of the algorithm to determine computational hot-spots. Implementations in C allow for the use of certain CPU profiling tools, described in Section \ref{subsubsection:vtune}, unlike higher-level languages such as Python. \\
    \hline
    Models Drone as Point   & In reality, implementing \gls{RRT} for a drone would model the robot as a \gls{3D} object defined by coordinates $\{x, y, z\}$ and Euler angles $\{\alpha, \beta, \gamma \}$. However, for simplicity's sake, modelling the drone as a point defined by coordinates $\{x, y, z\}$ will suffice. Time permitting, this could be revisited. \todo[inline]{Change this based on whether time does permit} \\
    \hline
    Mirrors Algorithm       & In order for the results of CPU performance analysis to be easy to understand, software implementation of \gls{RRT} should call functions that mirror the functions described in Algorithms \ref{algorithm:rrt} and \ref{algorithm:rrt_collision}. \\
    \hline
\end{tabular}
\caption{Technical Specifications for \gls{RRT} Implementation}
\label{table:RRT_Tech_Specs}
\end{center}
\end{table}
\todo[inline]{Improve this table}

    The original intention was to find an existing implementation of RRT that could fulfill these requirements. However, no open-source implementations were suitable. Appendix \ref{section:rrt_appendix_existing_implementations} shows an evaluation of existing implementations.

    As a result, it was necessary to build a C implementation of RRT from the ground up to the aformentioned specifications.

\subsection{Implementation Design}
    
    The design and implementation of \gls{RRT}, while neccessary, was significant and time consuming. Since this was not the main object of this thesis, only a brief description of key design choices has been included here. Appendix \ref{appendix:rrt_appendix} contains a more detailed account.

    \subsubsection{Parameterization}
        Table \ref{table:rrt_params} shows the parameters that were included in the implementation and compiled by way of a C header file.
        % @Author: AnthonyKenny98
% @Date:   2020-04-05 17:54:44
% @Last Modified by:   AnthonyKenny98
% @Last Modified time: 2020-04-05 18:10:00
\begin{table}[H]
\begin{centering}
\begin{tabular}{|c|c|l|}
\hline
\textbf{Parameter} & \textbf{Data Type} & \textbf{Description} \\
\hline
$\epsilon$ & Integer & Maximum distance between two configurations \\
\hline
$K$ & Integer & Maximum number of configurations in the graph \\
\hline
$DIM$ & Integer & Upper bound of each axis of workspace \\
\hline
Goal Bias & Float & Percentage likelihood of stepping towards goal node \\
\hline
\gls{OGM} & File Pointer & \glsname{CSV} of booleans to represent grids \\
\hline
\end{tabular}
\caption{RRT Implementation Parameters}
\label{table:rrt_params}

\end{centering}
\end{table}

    \subsubsection{Key Functions}
        Algorithm \ref{algorithm:rrt_collision} shows that there are 5 key functions that constitute \gls{RRT}. Figure \ref{fig:rrt_functions} demonstrates each of these functions: \texttt{getRandomConfig()}, \texttt{findNearestConfig()}, \texttt{stepFromNearest()}, \texttt{(configCollisions)}, and \texttt{edgeCollisions()}.
        % @Author: AnthonyKenny98
% @Date:   2020-04-05 18:33:38
% @Last Modified by:   AnthonyKenny98
% @Last Modified time: 2020-04-05 18:43:04
\begin{figure}[H]
\begin{centering}
\begin{tabular}{ccc}

    \begin{subfigure}{0.3\linewidth}
    \includegraphics[width=\linewidth]{chapters/chapter2/img/keyfunctions/functions1.png}
    \caption{}
    \end{subfigure} & 

    \begin{subfigure}{0.3\linewidth}
    \includegraphics[width=\linewidth]{chapters/chapter2/img/keyfunctions/functions2.png}
    \caption{}
    \end{subfigure} &

    \begin{subfigure}{0.3\linewidth}
    \includegraphics[width=\linewidth]{chapters/chapter2/img/keyfunctions/functions3.png}
    \caption{}
    \end{subfigure} \\

    \begin{subfigure}{0.3\linewidth}
    \includegraphics[width=\linewidth]{chapters/chapter2/img/keyfunctions/functions4.png}
    \caption{}
    \end{subfigure} &

    \begin{subfigure}{0.3\linewidth}
    \includegraphics[width=\linewidth]{chapters/chapter2/img/keyfunctions/functions5.png}
    \caption{}
    \end{subfigure} & 

    \begin{subfigure}{0.3\linewidth}
    \includegraphics[width=\linewidth]{chapters/chapter2/img/keyfunctions/functions6.png}
    \caption{}
    \end{subfigure} \\

\end{tabular}
\caption{}
\label{fig:rrt_functions}
\end{centering}
\end{figure}

    \subsubsection{Dimensionality}

    \subsubsection{$K$:DIM Ratio}

\subsection{Visualizing Implementation}
% @Author: AnthonyKenny98
% @Date:   2020-02-23 14:14:12
% @Last Modified by:   AnthonyKenny98
% @Last Modified time: 2020-03-01 08:06:12


\begin{figure}[H]
\begin{center}
\begin{tabular}{c  c}
    \includegraphics[width=0.45\linewidth]{chapters/chapter2/img/rrt_2d_1.png} & \includegraphics[width=0.45\linewidth]{chapters/chapter2/img/rrt_2d_2.png} \\
    \includegraphics[width=0.45\linewidth]{chapters/chapter2/img/rrt_2d_3.png} & \includegraphics[width=0.45\linewidth]{chapters/chapter2/img/rrt_2d_4.png}
    \end{tabular}
    \caption{2D RRT Implementation shown by \ac{GUI}}
    \label{figure:2DrrtGui}
\end{center}
\end{figure}

\todo[inline]{Describe implementation in 3D}
% @Author: AnthonyKenny98
% @Date:   2020-02-23 14:14:12
% @Last Modified by:   AnthonyKenny98
% @Last Modified time: 2020-02-23 14:19:03


\begin{figure}[H]
\begin{center}
    % \begin{tabular}{c  c}
    % \includegraphics[draft=false,width=0.45\linewidth]{chapters/chapter2/img/rrt_2d_1.png} & \includegraphics[draft=false,width=0.45\linewidth]{chapters/chapter2/img/rrt_2d_2.png} \\
    % \includegraphics[draft=false,width=0.45\linewidth]{chapters/chapter2/img/rrt_2d_3.png} & \includegraphics[draft=false,width=0.45\linewidth]{chapters/chapter2/img/rrt_2d_4.png}
    % \end{tabular}
    \missingfigure[figwidth=\linewidth, figheight=10cm]{Simulations of 3D RRT Implementations}
    \caption{3D RRT Implementation shown by \ac{GUI}}
    \label{figure:3DrrtGui}
\end{center}
\end{figure}


\newpage
\section{Performance Analysis}
\label{section:rrt_analysis}
    % @Author: AnthonyKenny98
% @Date:   2020-02-28 15:02:19
% @Last Modified by:   AnthonyKenny98
% @Last Modified time: 2020-03-01 15:49:31

\todo[inline]{Brief introduction outlining purpose of performance analysis}

\subsection{Methodology}
    To restate, the aim of this thesis is to design a computer processor with reduced execution time of motion planning algorithms, such as \ac{RRT}. As such, it is important to understand the elements of the algorithm that have the highest percentage of CPU execution time. To determine this, it was necessary to implement my own, naive but typical, \ac{RRT} in C. This program could then be compiled and analysed using a software performance profiling tool. With this, I could design experiments to determine the critical RRT functions (those occupying a majority of CPU time) and see how this varies given different parameters.
    \todo[inline]{Outline of method of analysis. Something better than the above}

    \subsubsection*{VTune Profiler}
    \label{subsubsection:vtune}
        VTune Profiler performance profiler is an application for software performance analysis. It provides functionality to examine hot-spots for CPU execution time through a top down analysis, shown below in Figure \ref{figure:VTuneTopDown}. As can be seen from the figure, the top down analysis tool shows the percentage of CPU time taken up by each function. I used this tool to profile the algorithm's performance as I changed certain parameters.
        \todo[inline]{Rewrite the above}
        % @Author: AnthonyKenny98
% @Date:   2020-02-23 14:33:19
% @Last Modified by:   AnthonyKenny98
% @Last Modified time: 2020-02-23 14:36:06

\begin{figure}[H]
\begin{center}
    \missingfigure[figwidth=\linewidth]{Screenshot of VTune Top Down Analysis (Maybe)}
    \caption{VTune Amplifier TopDown Analysis Example}
    \label{figure:VTuneTopDown}
\end{center}
\end{figure}

    \subsubsection*{Internal Timing}
        The limitation of VTune Profiler is that it can only profile software running on Intel processors, which implement the x86-64 \ac{ISA}. As such, when the time comes to analyse performance of the software running on a RISC-V processor, another method will be required. A simple and effective way of measuring execution performance is to insert timing functionality into the software itself. \\

        \todo[inline]{Provide or link to appendix of explanation of internal timing}

    \subsubsection*{Comparison}
        Before proceeding to use either of these methods to profile the software implementation of \ac{RRT}, it was important to verify that the two methods yielded similar results for the same program. Table \ref{table:timing_calibration} summarizes the results of analysis of a simple C executable. The program calls 5 functions, $\{A, B, C, D, E\}$, each a simple iteration in which a integer is incremented. Since the Internal Timing method returned similar results to the (trusted) VTune Profiler, it was considered to be a reliable method. While it was encouraging to see both methods returned similar results for absolute execution time, the more important metric was the similarity in percentage of total execution time.

        \begin{table}[H]
\begin{center}
\begin{tabular}{|m{0.1\linewidth}|m{0.18\linewidth}|m{0.18\linewidth}|m{0.18\linewidth}|m{0.18\linewidth}|}
\hline
\multirow{2}{*}{function} & \multicolumn{2}{c|}{Vtune Profiler} & \multicolumn{2}{c|}{Internal Timing} \\
\cline{2-5}
            & time (s)  & time (\% total) & time (s) & time (\% total)    \\
\hline
A       & 0.488     & 57.4\%        & 0.497 & 57.6\%           \\
B       & 0.2       & 23.5\%        & 0.198 & 23.1\%           \\
C       & 0.102     & 12.0\%        & 0.099 & 11.5\%           \\
D       & 0.048     & 5.7\%         & 0.049 & 5.6\%            \\
E       & 0.012     & 1.4\%         & 0.019 & 2.2\%            \\
\hline
\end{tabular}
\end{center}
\caption{Comparison of Timing Methods}
\label{table:timing_calibration}
\end{table}

    \subsubsection*{Experimental Design}
        In profiling \ac{RRT} in software, the goal was to find the critical task across different values of $K$ and sizes of configuration space. Multiple tests were run, varying these two constraints, to find this critical function. The results of this analysis can be found in Section \ref{section:rrt_analysis_results}.

\subsection{Results}
\label{section:rrt_analysis_results}
    Figure \ref{fig:rrt_profiling} shows the profile of functions within \ac{RRT}, for $100 \leq K \leq 10000$, and cubic configuration spaces with dimensions $\{4, 8, 16, 32\}$. Each subfigure shows a similar profile, with the \% of CPU Execution Time taken by findNearestNode increasing with $K$. This is to be expected. \todo{Explanation of time complexity}However, it is also seen that edgeCollisions increases with larger configuration spaces, taking up the overwhelming majority of execution time for a 32x32x32 configuration space.
    
    \newpage
    % @Author: AnthonyKenny98
% @Date:   2020-02-29 14:51:44
% @Last Modified by:   AnthonyKenny98
% @Last Modified time: 2020-04-03 14:28:30
\begin{figure}[H]
\begin{center}

    % Subfigure A
    \begin{subfigure}{\textwidth}
    \begin{center}
    \includegraphics[width=\linewidth,height=0.3\paperheight]{chapters/chapter2/img/profiling/4x4x4/performance.png}
    \caption{4x4x4 \gls{configuration} Space}
    \label{subfig:4x4x4rrt}
    \end{center}
    \end{subfigure}
    % 
    % Subfigure B
    \begin{subfigure}{\textwidth}
    \begin{center}
    \includegraphics[width=\linewidth,height=0.3\paperheight]{chapters/chapter2/img/profiling/8x8x8/performance.png}
    \caption{8x8x8 \gls{configuration} Space}
    \label{subfig:8x8x8rrt}
    \end{center}
    \end{subfigure}
    % Caption and Label
    \caption{\gls{RRT} Functions as a \% of Total CPU Exectution Time}
\end{center}
\end{figure}

\newpage
\begin{figure}[H]\ContinuedFloat
\begin{center}
    % 
    % Subfigure C
    \begin{subfigure}{\textwidth}
    \begin{center}
    \includegraphics[width=\linewidth,height=0.3\paperheight]{chapters/chapter2/img/profiling/16x16x16/performance.png}
    \caption{16x16x16 \gls{configuration} Space}
    \label{subfig:16x16x16rrt}
    \end{center}
    \end{subfigure}
    % 
    % Subfigure D
    \begin{subfigure}{\textwidth}
    \begin{center}
    \includegraphics[width=\linewidth,height=0.3\paperheight]{chapters/chapter2/img/profiling/32x32x32/performance.png}
    \caption{32x32x32 \gls{configuration} Space}
    \label{subfig:32x32x32rrt}
    \end{center}
    \end{subfigure} 
    
    % Caption and Label
    \caption{\gls{RRT} Functions as a \% of Total CPU Exectution Time (cont.)}
    \label{fig:rrt_profiling}
\end{center}
\end{figure}
\todo[inline]{Change Y axis to \% and increase text size}

    Furthermore, the computational load of findNearestNode can be reduced through a variety of software optimizations. A simple one used here to demonstrate that fact is storing nodes in seperate ``buckets,'' sorted by their $x$ value. By using only two buckets, the execution time of findNearestNode fell drastically. Figure \ref{subfig:32x32x32rrt2} shows edge collision detection accounting for over 95\% of execution time for $100 \leq K \leq 10000$. This is consistent with the profiling results of \ac{RRT} in prior work\cite{Bialkowski2011}.

    \subsubsection*{Conclusion}
        From the above data, it was identified that, as prior work suggested, edge collision detection shows the greatest promise for potential speedup through specialized hardware. The next chapter details the process of designing and building this hardware.
        \todo[inline]{Add simulations to determine correct K}

    % @Author: AnthonyKenny98
% @Date:   2020-02-29 17:30:44
% @Last Modified by:   AnthonyKenny98
% @Last Modified time: 2020-03-01 08:08:09
\begin{figure}[H]
\begin{center}

    % Subfigure A
    \begin{subfigure}{\textwidth}
    \begin{center}
    \includegraphics[width=\linewidth,height=0.3\paperheight]{chapters/chapter2/img/profiling/16x16x16/performance2.png}
    \caption{16x16x16 Configuration Space}
    \label{subfig:16x16x16rrt2}
    \end{center}
    \end{subfigure}
    % 
    % Subfigure B
    \begin{subfigure}{\textwidth}
    \begin{center}
    \includegraphics[width=\linewidth,height=0.3\paperheight]{chapters/chapter2/img/profiling/32x32x32/performance2.png}
    \caption{32x32x32 Configuration Space}
    \label{subfig:32x32x32rrt2}
    \end{center}
    \end{subfigure}
    % Caption and Label
    \caption{\ac{RRT} Functions Exectution Time, with Bucket Optimization}
    \label{fig:rrt_profiling2}
\end{center}
\end{figure}
