\begin{table}[H]
\begin{center}
\begin{tabular}{|p{.3\linewidth}|p{.64\linewidth}|}
    \hline
    Requirement             & Description and Justification \\
    \hline
    C/C++ Implementation    & As outlined in Section \ref{subsection:project_structure}, the critical step in determining the design of specialized hardware to accelerate \gls{RRT} is CPU performance analysis of the algorithm to determine computational hot-spots. Implementations in C allow for the use of certain CPU profiling tools, described in Section \ref{subsubsection:vtune}, unlike higher-level languages such as Python. \\
    \hline
    Models Drone as Point   & In reality, implementing \gls{RRT} for a drone would model the robot as a \gls{3D} object defined by coordinates $\{x, y, z\}$ and Euler angles $\{\alpha, \beta, \gamma \}$. However, for simplicity's sake, modelling the drone as a point defined by coordinates $\{x, y, z\}$ will suffice. Time permitting, this could be revisited. \todo[inline]{Change this based on whether time does permit} \\
    \hline
    Mirrors Algorithm       & In order for the results of CPU performance analysis to be easy to understand, software implementation of \gls{RRT} should call functions that mirror the functions described in Algorithms \ref{algorithm:rrt} and \ref{algorithm:rrt_collision}. \\
    \hline
\end{tabular}
\caption{Technical Specifications for \gls{RRT} Implementation}
\label{table:RRT_Tech_Specs}
\end{center}
\end{table}
\todo[inline]{Improve this table}