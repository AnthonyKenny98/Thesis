\begin{table}[H]
\begin{center}
\begin{tabular}{|p{.2\linewidth}|p{.74\linewidth}|}
    \hline
    Requirement             & Brief Description and Justification \\
    \hline
    Implemented in C/C++    & 
        Implementations in C allow for more accurate analysis of computational bottlenecks, unlike higher-level languages like Python. \\
    \hline
    3D Workspace            & 
        The computational requirements of \gls{RRT} in \gls{3D} differ somewhat to that in \gls{2D}. Since autonomous \glspl{UAV} operate in 3D space, it was neccesary to have a \gls{3D} implementation to analyse. \\
    \hline
    \Gls{UAV} modelled as a \gls{3D} rectangular prism  & 
        In theory, it is possible to model a \gls{UAV} much more precisely than a rectangular prism. However, in reality, modelling a \gls{UAV} as a \gls{3D} rectangular prism, defined by coordinates $\{x, y, z\}$ and Euler angles $\{\alpha, \beta, \gamma \}$, is more than sufficient (and more computationally efficient). See Appendix \ref{section:rrt_appendix_modelling} for justification. \\
    \hline
    Mathematically Complete Collision Detection & 
        When \gls{RRT} is implemented for educational purposes, the edge collision calulations are often simplified to a sampling model which is \gls{probabilistically complete} but not \gls{mathematically complete}. In other words, it will catch most collisions by sampling a number of points along each edge, but there is always a possibility of an undetected collision. In real world applications, collisions must be calculated by method of geometric intersection to ensure all collisions are detected. \\
    \hline
    Highly Parameterizable      & 
        Accurate analysis of the algorithm required the ability to vary the following parameters: 
        \begin{itemize}
        \item $\epsilon$ (Maximum distance between two configurations)
        \item $K$ (Maximum number of configurations)
        \item $DIM$ (The upper bound of each dimension for a $DIM\times DIM\times DIM$ workspace)
        \item Goal Bias (How biased RRT is to move towards goal point)
        \end{itemize} \\
    \hline
\end{tabular}
\caption{Abbreviated Technical Specifications for \gls{RRT} Implementation}
\label{table:RRT_Tech_Specs_Abbrev}
\end{center}
\end{table}
