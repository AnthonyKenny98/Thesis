% @Author: AnthonyKenny98
% @Date:   2020-04-05 14:09:51
% @Last Modified by:   AnthonyKenny98
% @Last Modified time: 2020-04-11 03:02:28

\begin{table}[H]
\begin{center}
\begin{tabular}{|p{.2\linewidth}|p{.74\linewidth}|}
    \hline
    \multicolumn{2}{|c|}{\textbf{General Specifications}} \\
    \hline
    \textbf{Requirement}             & \textbf{Description and Justification} \\
    \hline
    Implemented in C/C++    & 
        As outlined in Section \ref{subsection:project_structure}, the critical step in determining the design of specialized hardware to accelerate \gls{RRT} is CPU performance analysis of the algorithm to determine computational hot-spots. Implementations in C allow for the use of certain CPU profiling tools, unlike higher-level languages such as Python. \\
    \hline
    3D Workspace            & 
        The computational requirements of \gls{RRT} in \gls{3D} differs somewhat to that in \gls{2D}. Since autonomous \glspl{UAV} operate in 3D space, it was neccesary to have a \gls{3D} implementation to analyse. \\
    \hline
    \Gls{UAV} modelled in \gls{3D} as a rectangular prism  & 
        In theory, it is possible to model a \gls{UAV} much more precisely than a rectangular prism, taking into account its shape and negative space. However, in reality, modelling a \gls{UAV} as a \gls{3D} rectangular prism, defined by coordinates $\{x, y, z\}$ and Euler angles $\{\alpha, \beta, \gamma \}$, is more than sufficient (and more efficient). See Appendix \ref{section:rrt_appendix_modelling} for justification of this. \\
    \hline
    Mathematically Complete Collision Detection & 
        When \gls{RRT} is implemented for educational purposes, the edge collision calulations are often simplified to a sampling model which is \gls{probabilistically complete} but not \gls{mathematically complete}. In other words, it will catch most collisions by sampling a number of points along each edge, but there is always a possibility of an undetected collision. In real world applications, collisions must be calculated by method of geometric intersection to ensure all collisions are detected. \\
    \hline
\end{tabular}
\mycaption{General Technical Specifications for \gls{RRT} Implementation}{}
\label{table:RRT_Tech_Specs_General}
\end{center}
\end{table}

\begin{table}[H]
\begin{center}
\begin{tabular}{|p{.2\linewidth}|p{.74\linewidth}|}
    \hline
    \multicolumn{2}{|c|}{\textbf{Required Parameters}} \\
    \hline
    \textbf{Parameter}   & \textbf{Description and Justification} \\
    \hline
    $\epsilon$ (a.k.a. $\Delta q$) & 
        The maximum difference between two configurations. Larger values of $\epsilon$ can solve less obstacle dense problems faster, but take longer to solve problems with tight corners.\\
    \hline
    $K$ &
        The maximum number of configurations. This is largely correlative to the amount of time the user will allow the algorithm to run. Larger values of $K$ will take longer but generate better paths, while smaller values will execute for less time but generate more jagged paths or may not reach the goal node. The value of $K$ was varied to find the minimum execution time while still reaching the goal with high probability. \\
    \hline
    $DIM$ &
        The upper bound of each axis of a $DIM\times DIM\times DIM$ Workspace. Larger values leave more space to be explored, and thus require larger values of $K$ to reach the goal with high likelihood. \\
    \hline
    Goal Bias &
        The given probability that the graph will extend the graph $\epsilon$ distance from an existing configuration to a new configuration in the direction of the goal. \\
    \hline
\end{tabular}
\mycaption{Required Parameters for \gls{RRT} Implementation}{}
\label{table:RRT_Tech_Specs_Parameters}
\end{center}
\end{table}