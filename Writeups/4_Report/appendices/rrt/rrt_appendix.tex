% @Author: AnthonyKenny98
% @Date:   2020-03-19 12:56:13
% @Last Modified by:   AnthonyKenny98
% @Last Modified time: 2020-04-06 10:20:38

\section{Justification of Modelling UAV as Prism}
\label{section:rrt_appendix_modelling}
    While it is possible for a \gls{UAV} to be modelled in precise detail, taking into account its exact shape, more often \glspl{UAV} are modelled as a 3D prism in motion planning problems, for the following reasons:
    \begin{itemize}
    \item It is a rare case that the negative space gained by modelling in such detail is utilised
    \item Representation of drone's configuration is much more complex.
    \item Computing edge collisions is much more computationally intensive.
    \end{itemize}

    % @Author: AnthonyKenny98
% @Date:   2020-04-05 12:55:29
% @Last Modified by:   AnthonyKenny98
% @Last Modified time: 2020-04-05 15:43:13

\begin{figure}[H]
\begin{centering}

\begin{tabular}{cc}
\begin{subfigure}{0.45\linewidth}
    \includegraphics[width=\linewidth]{appendices/rrt/img/DroneNegSpace.png}
    \caption{}
    \label{subfig:dronenegspace}
\end{subfigure}
\begin{subfigure}{0.45\linewidth}
    \includegraphics[width=\linewidth]{appendices/rrt/img/DroneRecPrism.png}
    \caption{}
    \label{subfig:dronerecprism}
\end{subfigure}
\end{tabular}
\caption[Modelling a UAV as a Rectangular Prism]{Modelling a \gls{UAV} as a Rectangular Prism. Red highlight demonstrates the model, overlayed over the exact schematic. Figure \ref{subfig:dronenegspace} shows how a drone can be modelled in high detail, but gains little useful free space when compared with Figure \ref{subfig:dronerecprism}, which models a drone as a rectangular prism.\cite{Thingbits}}
\label{fig:dronemodelling}
\end{centering}
\end{figure}

\section{Full Technical Specifications for RRT Implementation}
\label{section:rrt_appendix_tech_specs}
    % @Author: AnthonyKenny98
% @Date:   2020-04-05 14:09:51
% @Last Modified by:   AnthonyKenny98
% @Last Modified time: 2020-04-11 03:02:28

\begin{table}[H]
\begin{center}
\begin{tabular}{|p{.2\linewidth}|p{.74\linewidth}|}
    \hline
    \multicolumn{2}{|c|}{\textbf{General Specifications}} \\
    \hline
    \textbf{Requirement}             & \textbf{Description and Justification} \\
    \hline
    Implemented in C/C++    & 
        As outlined in Section \ref{subsection:project_structure}, the critical step in determining the design of specialized hardware to accelerate \gls{RRT} is CPU performance analysis of the algorithm to determine computational hot-spots. Implementations in C allow for the use of certain CPU profiling tools, unlike higher-level languages such as Python. \\
    \hline
    3D Workspace            & 
        The computational requirements of \gls{RRT} in \gls{3D} differs somewhat to that in \gls{2D}. Since autonomous \glspl{UAV} operate in 3D space, it was neccesary to have a \gls{3D} implementation to analyse. \\
    \hline
    \Gls{UAV} modelled in \gls{3D} as a rectangular prism  & 
        In theory, it is possible to model a \gls{UAV} much more precisely than a rectangular prism, taking into account its shape and negative space. However, in reality, modelling a \gls{UAV} as a \gls{3D} rectangular prism, defined by coordinates $\{x, y, z\}$ and Euler angles $\{\alpha, \beta, \gamma \}$, is more than sufficient (and more efficient). See Appendix \ref{section:rrt_appendix_modelling} for justification of this. \\
    \hline
    Mathematically Complete Collision Detection & 
        When \gls{RRT} is implemented for educational purposes, the edge collision calulations are often simplified to a sampling model which is \gls{probabilistically complete} but not \gls{mathematically complete}. In other words, it will catch most collisions by sampling a number of points along each edge, but there is always a possibility of an undetected collision. In real world applications, collisions must be calculated by method of geometric intersection to ensure all collisions are detected. \\
    \hline
\end{tabular}
\mycaption{General Technical Specifications for \gls{RRT} Implementation}{}
\label{table:RRT_Tech_Specs_General}
\end{center}
\end{table}

\begin{table}[H]
\begin{center}
\begin{tabular}{|p{.2\linewidth}|p{.74\linewidth}|}
    \hline
    \multicolumn{2}{|c|}{\textbf{Required Parameters}} \\
    \hline
    \textbf{Parameter}   & \textbf{Description and Justification} \\
    \hline
    $\epsilon$ (a.k.a. $\Delta q$) & 
        The maximum difference between two configurations. Larger values of $\epsilon$ can solve less obstacle dense problems faster, but take longer to solve problems with tight corners.\\
    \hline
    $K$ &
        The maximum number of configurations. This is largely correlative to the amount of time the user will allow the algorithm to run. Larger values of $K$ will take longer but generate better paths, while smaller values will execute for less time but generate more jagged paths or may not reach the goal node. The value of $K$ was varied to find the minimum execution time while still reaching the goal with high probability. \\
    \hline
    $DIM$ &
        The upper bound of each axis of a $DIM\times DIM\times DIM$ Workspace. Larger values leave more space to be explored, and thus require larger values of $K$ to reach the goal with high likelihood. \\
    \hline
    Goal Bias &
        The given probability that the graph will extend the graph $\epsilon$ distance from an existing configuration to a new configuration in the direction of the goal. \\
    \hline
\end{tabular}
\mycaption{Required Parameters for \gls{RRT} Implementation}{}
\label{table:RRT_Tech_Specs_Parameters}
\end{center}
\end{table}

\section{Assessment of Existing RRT Implementations}
\label{section:rrt_appendix_existing_implementations}
    % @Author: AnthonyKenny98
% @Date:   2020-04-05 13:32:41
% @Last Modified by:   AnthonyKenny98
% @Last Modified time: 2020-04-05 13:46:28

\begin{table}[H]
\begin{centering}
\begin{tabular}{|p{0.2\linewidth}|p{0.15\linewidth}|p{0.15\linewidth}|p{0.15\linewidth}|p{0.15\linewidth}|}
\hline
Repository  & Language  &  Workspace Dimension  & Object Model & Algorithmic Correctness \\
\hline
RoboJackets\cite{RoboJackets2019}   & \textcolor{mygreen}{C++}    &       &       & \\
\hline
\end{tabular}
\caption[Evaluation of Existing Open-Source Implementations of RRT]{Evaluation of Existing Open-Source Implementations of RRT. Links to Github repositories can be found in the Bibliography.}
\end{centering}
\end{table}

\section{Implementation of Key RRT Functions}
\label{section:rrt_appendix_function_impl}
    % @Author: AnthonyKenny98
% @Date:   2020-04-05 21:30:06
% @Last Modified by:   AnthonyKenny98
% @Last Modified time: 2020-04-11 03:04:30

\bigskip
\begin{algorithm}[H]
    \caption{\texttt{getRandomConfig()} as implemented for \gls{RRT}}
    \SetAlgoLined
    \SetArgSty{textnormal}
    \begin{tabular}{l l}
    \textbf{Inputs:}    & Dimensionality $N$,\\ 
                        & Upper Axis Bound $DIM$ \\
    \textbf{Output:}    & Random Configuration $q$ \\
    \end{tabular}

        $q.x \leftarrow$ randomFloat($DIM$) \\
        $q.y \leftarrow$ randomFloat($DIM$) \\
        $q.\alpha \leftarrow$ randomFloat($2\pi$) \\
        \If{$N == 3$}{
            $q.z \leftarrow$ randomFloat($DIM$) \\
            $q.\beta \leftarrow$ randomFloat($2\pi$) \\
            $q.\gamma \leftarrow$ randomFloat($2\pi$) \\
        }
        \Return $q$;\\
\end{algorithm}
\bigskip
Where \texttt{randomFloat(max)} returns a float between 0 and \texttt{max}.

\bigskip
\begin{algorithm}[H]
    \caption{\texttt{findNearestConfig()} as implemented for \gls{RRT}}
    \SetAlgoLined
    \SetArgSty{textnormal}
    \begin{tabular}{l l}
    \textbf{Inputs:}    & Graph $G$, \\
                        & New Configuration $q_{new}$ \\
    \textbf{Output:}    & Nearest Configuration $q_{nearest}$ \\
    \end{tabular}

        $q_{nearest} \leftarrow$ $G$.$q_{init}$ \\
        \For{$k = 0$ to $G$.existing\_nodes}{
            \If{distance($q_{new}$, $G.q$[$k$]) < distance($q_{new}$, $q_{nearest}$)}{
                $q_{nearest} \leftarrow$ $G.q$[$k$] \\
            }
        }
        \Return $q_{nearest}$ \\
\end{algorithm}
\bigskip
Where \texttt{distance($q_1$, $q_2$)} returns the Euclidean distance between two configurations.

\bigskip
\begin{algorithm}[H]
    \caption{\texttt{stepFromNearest()} as implemented for \gls{RRT}}
    \SetAlgoLined
    \SetArgSty{textnormal}
    \begin{tabular}{l l}
    \textbf{Inputs:}    & Configuration in Graph $q_{nearest}$,\\ 
                        & New Configuration $q_{new}$, \\
                        & Goal Bias $B$, \\
                        & Maximum Step Distance $\epsilon$, \\
                        & Graph $G$ \\
    \textbf{Output:}    & Updated New Configuration $q_{new}$ \\
    \end{tabular}

        \If{distance($q_{nearest}$, $q_{new}$) > $\epsilon$}{
            \If{randomFloat($1$) < $B$}{
                $q_{new} \leftarrow $ stepTowardConfig($q_{nearest}$, $G$.$q_{goal}$) \\
            } \Else{
                $q_{new} \leftarrow $ stepTowardConfig($q_{nearest}$, $q_{new}$) \\
            }
        }
        \Return $q_{new}$;\\
\end{algorithm}
\bigskip
Where \texttt{stepTowardConfig($q_1$, $q_2$)} returns a configuration $\epsilon$ from $q_1$ in the direction of $q_2$.

\bigskip
\begin{algorithm}[H]
    \caption{\texttt{configCollision()} as implemented for \gls{RRT}}
    \SetAlgoLined
    \SetArgSty{textnormal}
    \begin{tabular}{l l}
    \textbf{Inputs:}    & Dimensionality $N$,\\ 
                        & Occupancy Grid Map ($N$-Dimensional Array) $O$,\\ 
                        & Configuration $q$ \\
    \textbf{Output:}    & Boolean \\
    \end{tabular}

        \If{$N$ == 2 }{
            \Return $O$[gridLookup($q.x$)][gridLookup($q.y$)]
        }
        \Else{
            \Return $O$[gridLookup($q.x$)][gridLookup($q.y$)][gridLookup($q.z$)]
        }
\end{algorithm}
\bigskip
Where $O$ is a $N$-Dimensional array of booleans, with True representing an occupied grid and false representing an unoccupied one. \texttt{gridLookup()} is a function that maps a floating point coordinate to the correct integer of the grid in which it resides. For a map resolution of one, this is as simple as rounding a float down to an integer. \\

\begin{algorithm}[ht!]
    \caption{\texttt{configCollision()} as implemented for \gls{RRT} for 3D}
    \SetAlgoLined
    \SetArgSty{textnormal}
    \begin{tabular}{l l}
    \textbf{Inputs:}    & Edge $e$,\\ 
                        & Occupancy Grid Map (3-Dimensional Array) $O$,\\ 
                        & Maximum Step Distance $\epsilon$ \\
    \textbf{Output:}    & Boolean \\
    \end{tabular}
    
        $q_{min} \leftarrow $ minConfig($e.q_1$, $e.q_2$) \\
        \For{($x=  q_{min}.x$ to $q_{min}.x + \epsilon$)}{
            $q_{intersection} \leftarrow$ = edgeIntersectsPlane($e$, $x$) \\
            \If{$O$[$q_{intersection}.x$][$q_{intersection}.y$][$q_{intersection}.z$]}{
                \Return true
            }
        }
        \For{($y=  q_{min}.y$ to $q_{min}.y + \epsilon$)}{
            $q_{intersection} \leftarrow$ = edgeIntersectsPlane($e$, $y$) \\
            \If{$O$[$q_{intersection}.x$][$q_{intersection}.y$][$q_{intersection}.z$]}{
                \Return true
            }
        }
        \For{($z=  q_{min}.z$ to $q_{min}.z + \epsilon$)}{
            $q_{intersection} \leftarrow$ = edgeIntersectsPlane($e$, $z$) \\
            \If{$O$[$q_{intersection}.x$][$q_{intersection}.y$][$q_{intersection}.z$]}{
                \Return true
            }
        }
        \Return false
\end{algorithm}

While seemingly complex, the above algorithm merely steps through the mathematical process of checking the relevant $x$, $y$, and $z$ planes for a point of intersection with the edge $e$. It then looks up the \gls{OGM} $O$ to see if the grid corresponding with the point of intersection is occupied. If so, then it reports a collision by returning True. The function \texttt{edgeIntersectsPlane} follows the geometrical process of detecting a segment-plane intersection outlined in Appendix \ref{section:rrt_appendix_line_plane_intersection}. $q_{min}$ is calculated to be the origin point of the grid closest to the origin. In other words, the algorithm does not check for intersections throughout the entire map, only the maximum number of grids that could possible be intersected by the edge $e$, given the location of the two points of the edge, $e.p_1$ and $e.p_2$, and the maximum edge length $\epsilon$. The algorithm for \texttt{edgeCollision()} in \gls{2D} can be inferred from the above, checking segment-line intersections for $x$ and $y$ lines.

\newpage
\section{Geometrically Determining Segment-Plane Intersection}
\label{section:rrt_appendix_line_plane_intersection}
    % @Author: AnthonyKenny98
% @Date:   2020-04-06 09:45:02
% @Last Modified by:   AnthonyKenny98
% @Last Modified time: 2020-04-06 11:28:04
The method of edge collision detection in this project's implementation of \gls{RRT} relies on detecting segment-plane intersections. The planes are always set up to be parallel with either the $x-y$ plane, the $x-z$ plane, or the $y-z$ plane. Figure \ref{fig:parallelPlanes} demonstrates this point.

\begin{figure}[H]
\begin{centering}
\includegraphics[width=\linewidth]{appendices/rrt/img/planes}
\caption{Using Parallel Planes to determine Edge Collisions with Grids}
\label{fig:parallelPlanes}
\end{centering}
\end{figure}

A plane can be defined by 3 points, $P_a$, $P_b$, and $P_c$. In practice, the points defining a plane parallel to the x-y plane would have the following points:

    $$P_a = (x,y,z)$$
    $$P_b = (x+\Delta x,y,z)$$
    $$P_c = (x,y+\Delta y,z)$$

Two vectors, $\vec{AB}$ and $\vec{AC}$ can be determined.

The normal to the plane is the cross product: 
    $$\vec{AB} \times \vec{AC}$$

And the equation of the plane written as:

    $$a(x-x_0) + b(y-y_0) + c(z-z_0) = 0$$
    $$ax + by + cz = ax_0 + by_0 + cz_0$$

Where $<a,b,c>$ is the normal to the plane and $(x_0, y_0, z_0)$ is one of the points $P_a$, $P_b$, or $P_c$.
The RHS can be set to equal $d$, leaving:

    $$ax + by + cz = d$$

Now, the equation of a line can be written in the form:

    $$ax + by + cz = 0$$

And can be parameterized in the following form:
    $$\begin{cases}
        x = & x_1 + t(x_2 - x_1) \\
        y = & y_1 + t(y_2 - y_1) \\
        z = & z_1 + t(z_2 - z_1) \\

    \end{cases}$$

To find the point of intersection, we substitute the equation of the line into the equation of the plane, yielding:

    $$a(x_1 + t(x_2 - x_1)) + b(x_1 + t(x_2 - x_1)) + c(x_1 + t(x_2 - x_1)) = d$$

Rearranging to find an expression for $t$:

    $$t = \frac{d - (ax_1 + by_1 + cz_1)}{a(x_2-x_1) + b(y_2-y_1) + c(z_2-z_1)}$$

Knowing $t$, we can find the point of intersection, $P_X$ to be:

    $$\begin{cases}
        x_X(t) = & x_1 + t(x_2 - x_1) \\
        y_X(t) = & y_1 + t(y_2 - y_1) \\
        z_X(t) = & z_1 + t(z_2 - z_1) \\
    \end{cases}$$

Finally, the following equalities are evaluated to see if the point lies on the segment:

    $$x_1 \leq x_X \leq x_2$$
    $$y_1 \leq y_X \leq y_2$$
    $$z_1 \leq z_X \leq z_2$$

If so, then the grids corresponding to the point of intersection can be marked as intersected.

\section{Timing Methodology of RRT Analysis}
\label{section:rrt_appendix_timing}
    % @Author: AnthonyKenny98
% @Date:   2020-04-06 08:00:17
% @Last Modified by:   AnthonyKenny98
% @Last Modified time: 2020-04-06 15:28:48

\subsubsection*{VTune Profiler}
    VTune Profiler is an application for software performance analysis. It provides functionality to examine hot-spots for CPU execution time through a top down analysis. The top down analysis tool shows the percentage of CPU time taken up by each function. It was used to initially profile the algorithm's performance.

    \subsubsection{Internal Timing}
        There are several limitations to using VTune Profiler. First, it can only profile software running on Intel processors, which implement the x86 \gls{ISA}. In anticipation of potentially needing to run performance analysis on a \gls{RISC-V} processor, another method was required. Secondly, VTune Profiler takes a long time to run, as it needs to conduct a lot of analysis that is extraneous to the purpose of this thesis. This became prohibitive when it came to conducting hundreds of tests for different parameterizations, with each test running \gls{RRT} a minimum of 100 times. Finally, it was not customizable to ignore certain parts of the implementation, such as logging functionality. While the implementation was designed in such a way that these should not intefere, it led to a lot of irrelevant data. A simple and effective alternative for measuring execution performance was to insert timing functionality into the software itself.

        Internal timing was implemented based on the inbuilt C \texttt{clock()} function and \\ \texttt{CLOCKS\_PER\_CYCLE} macro, and wrapping each function of interest in a performance tracking struct. This can be seen in the project's \gls{RRT} sub-repository under \texttt{performance.h}.

    \subsubsection*{Comparison}
        Before proceeding to use the internal timing method, it was important to verify that this method yielded similar results to VTune Profiler for the same program. Table \ref{table:timing_calibration} summarizes the results of analysis of a simple C executable. The program calls 5 functions, $\{A, B, C, D, E\}$, each a simple iteration in which a integer is incremented. Since the Internal Timing method returned similar results to the (trusted) VTune Profiler, it was considered to be a reliable method. While it was encouraging to see both methods returned similar results for absolute execution time, the more important metric was the similarity in percentage of total execution time. For good measure, a $\chi^2$ test of hypothesis was conducted and for one degree of freedom showed more that acceptable results.

        \begin{table}[H]
        \begin{center}
        \begin{tabular}{|m{0.09\linewidth}|m{0.16\linewidth}|m{0.16\linewidth}|m{0.16\linewidth}|m{0.16\linewidth}|m{0.09\linewidth}|}
        \hline
        \multirow{2}{*}{function} & \multicolumn{2}{c|}{Vtune Profiler} & \multicolumn{2}{c|}{Internal Timing}  & \multirow{2}{*}{$\chi^2$}\\
        \cline{2-5}
                    & time (s)  & time (\% total) & time (s) & time (\% total)  &   \\
        \hline
        A       & 0.488     & 57.4\%        & 0.497 & 57.6\%  &  0.00016       \\
        B       & 0.2       & 23.5\%        & 0.198 & 23.1\%  &  0.00002       \\
        C       & 0.102     & 12.0\%        & 0.099 & 11.5\%  &  0.00009       \\
        D       & 0.048     & 5.7\%         & 0.049 & 5.6\%   &  0.00002       \\
        E       & 0.012     & 1.4\%         & 0.019 & 2.2\%   &  0.00408       \\
        \hline
        \end{tabular}
        \end{center}
        \caption{Comparison of Timing Methods}
        \label{table:timing_calibration}
        \end{table}
\newpage
\section{Justification of K:DIM Ratio}