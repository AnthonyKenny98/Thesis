% @Author: AnthonyKenny98
% @Date:   2020-04-07 14:14:38
% @Last Modified by:   AnthonyKenny98
% @Last Modified time: 2020-04-09 08:45:54

\newpage
\section{Technical Specifications for Edge Collision Unit}
    \documentclass[11pt, oneside]{article}      % use "amsart" instead of "article" for AMSLaTeX format
\usepackage{geometry}                       % See geometry.pdf to learn the layout options. There are lots.
\geometry{letterpaper}                          % ... or a4paper or a5paper or ... 
%\geometry{landscape}                       % Activate for rotated page geometry
% \usepackage[parfill]{parskip}          % Activate to begin paragraphs with an empty line rather than an indent
\usepackage{graphicx}               % Use pdf, png, jpg, or eps§ with pdflatex; use eps in DVI mode
                                % TeX will automatically convert eps --> pdf in pdflatex        
\usepackage{amssymb}
\usepackage{amsmath}
\usepackage{array}
\usepackage{hyperref}
\hypersetup{
    colorlinks=true,
    citecolor=black,
    linkcolor=black,
    filecolor=magenta,      
    urlcolor=cyan,
}
\usepackage[printonlyused]{acronym}

%%%%%%%%%%%%%%%%%%%%%%%%%%%%%%%%%%%%%%%%%%%%%%%%%%%%%%%%%%

\title{\textsc{Technical Specifications} \\ 
	\small{for} \\ 
	\Large{RISC-V Architecture for Motion Planning Algorithms in Autonomous Drone Applications}}
\author{Anthony Kenny} 
\date{October 4th, 2019 \\ 
	Version: 1.0}

\begin{document}
\maketitle

\abstract{This thesis aims to design RISC-V computer architecture that supports fast execution of motion planning algorithms for drone applications. First, the computation of sampling-based motion planning algorithms commonly used in autonomous drones (such as RRT, RRT*, PRM) will be profiled on an unmodified RISC-V processor. From this profiling, common bottlenecks and hotspots in execution will be identified. Based on these results, this project will extend the RISC-V \ac{ISA} and design a modified processor to support the extensions.}

\section{Project Summary}


\subsection{Problem Statement}
Current processors cannot compute motion planning algorithms quickly enough for robots to operate in high complexity environments. Autonomous drones are a specific case of robots requiring real-time motion planning in complex environments. The state-of-the-art strategy of using a \ac{GPU} to accelerate the execution of these algorithms requires too much power to be cost effective or feasible for drones to sustain flight for useful periods of time.

\subsection{End User}
The end user of this project is a developer of autonomous drones. Such developers have a need for computing hardware that executes motion planning algorithms faster and more power efficiently than existing methods. This thesis will provide a processor design that is synthesizable on an \ac{FPGA}, giving developers a processer for which a \ac{RTOS}, or bare metal code, can be written. 
Additionally, these developers have a requirement that using a new processor for a drone will not require a massive investment in re-development. As such, this thesis will provide the toolchain necessary to compile C code into executable instructions on the new processor.

\subsubsection{Stakeholder Map}
Where the processor and \ac{ISA} lies within the map of drone, developer, user of drones.


\subsection{Project Requirements} \label{subsection:projectRequirements}
Table ~\ref{table:projectRequirements} outlines conceptually the requirements of the project. Specific technical specifications will be detailed in later sections. 

\begin{table}[h]
\begin{centering}
\begin{tabular}{| m{0.25\linewidth} | m{0.75\linewidth} |}
\hline
\textbf{Requirement}       & \textbf{Description} \\
\hline
RISC-V Compliance   & RISC-V is an extendable \ac{ISA}. This project will extend the \ac{ISA} to add new instructions. The contstraint of RISC-V compliance means that the new ISA I define must follow RISC-V conventions, and that the processor can implement any program compiled into the original RISC-V ISA, as well as the extended RISC-V ISA.\\
\hline
Synthesizable       & The finished processor design must be such that it is practically useable by drone developers. Since having such a processor design mass-produced on a chip is beyond the scope of this project, this project must deliver a processor defined in an \ac{HDL} that is synthesizable on an FPGA.\\
\hline
Speed               & One of the motivating factors of this project is the need for motion planning algorithms to execute faster for autonomous drones to become more useful in real world applications. \\
\hline
Power consumption   & The second motivating factor is the need for computation aboard drones to be as power efficient as possible to enable them to remain in flight for long enough periods of time.\\
\hline
\end{tabular}
\caption{Conceptual Outline of Project Requirements}
\label{table:projectRequirements}
\end{centering}
\end{table}

\subsection{Project Goals} \label{subsection:projectGoals}
This thesis aims to design a RISC-V processor, optimized for motion planning computation, that is synthesizeable on an \ac{FPGA} and adheres to the requirements outlined in Section \ref{subsection:projectRequirements}. It will also provide the tools necessary to complile programs for the processor. 

The nature of research into accelerating computation through modified computer architecture is such that, when asked a question of how fast/efficient/small/etc a system must be, the answer is, with consideration to certain trade-offs, as fast/efficient/small/etc as it can be! Thus, when defining goals for certain metrics such as speed or power efficiency, this thesis will do so by comparing performance of the modified processor with benchmark performance of an unmodified, off-the-shelf RISC-V processor synthesized on the same \ac{FPGA}.

\section{System Model Diagram}

\clearpage  


\section{Overall System Specifications}
Let the overall system be defined as 2 part: the Processor Module and the Compiler/Assembler Module. These parts are highlighted in Figure X \textbf{Complete!!!} \\
The following subsections detail the technical specifications of the overall system. As stated in Section \ref{subsection:projectGoals}, system specifications based on comparison with benchmarks will be compared against the same programs running on an unmodified, off-the-shelf RISC-V processor synthesized on the same \ac{FPGA}.
\subsection{Speed}
\subsubsection{Quantitative Description}
This project aims to achieve a speedup of at least one order of magnitude (10 times) when compared to benchmark performance, in the execution of pure motion planning algorithm programs. That is, given a program that implements, in C, an algorithm often used in autonomous drone motion planning, this project's system can return a valid path at least 10 times faster than a generic RISC-V processor. 

\subsubsection{Justification}

\subsubsection{Measurement}
There will be two broad stages of measurement for this metric. First, in simulation, the Vivado Design Suite\cite{Vivado} will allow the execution of a given compiled program to be timed on a simulated processor that is defined in an \ac{HDL}. Secondly, in synthesis, a to-be-determined tool will allow for me to time the execution of the same program, now on a processor physically synthesized on an FPGA. 

\subsection{Power Consumption}
\subsubsection{Quantitative Description}
Power is defined as energy dissipated over time. As such, when considering the application of this system in autonomous drones, we want to minimize the amount electrical energy committed to the computation of paths. Since the primary goal of this thesis is to reduce the execution time, it can aim to keep power use comparable between the benchmark system and the new system. If power remains roughly constant, but the time taken to execute a program is reduced 10 times, we should see a proportional improvement in energy efficiency.
\subsubsection{Justification}
\subsubsection{Measurement}
\clearpage




\section{Processor Specifications}
The first subcomponent of the system is the Processor Unit. It has two specifications, RISC-V Compliance and Syntheizability.

\subsection{RISC-V Compliance} \label{subsection:riscvCompliance}
\subsubsection{Quantitative Description}
A processor must be such that it supports the execution of all instructions defined in the \ac{ISA} for which it was designed. So too must this project's finished processor be able to correctly support any correctly compiled RISC-V assembly code. This may range from simple programs compiled into either the original or extended RISC-V \ac{ISA}, to complete operating systems, whether for drone applications or a generic linux distribution, for example. 
\subsubsection{Justification}

\subsubsection{Measurement}
Github\cite{Compliance} 

\subsection{Synthesizable}
\subsubsection{Quantitative Description}
The project must deliver a processor defined in an \ac{HDL} that is synthesizable on an FPGA for the project to be useful for drone developers. While this project will use and test with the Diligent Zync-7000 \ac{SoC}\cite{Zync}, the design should be synthesizable on most Zync boards.
\subsubsection{Justification}
\subsubsection{Measurement}
Measurement for this specification is relatively simple. Once the processor is designed in an \ac{HDL}, it can be synthesized onto an FPGA using the Vivado Design Suite, given the design is synthesizable (although this is not always simple to achieve). If it is not synthesizable, the design suite will throw and error.
Finally, while the design should be correct and RISC-V compliant by the tests performed in section \ref{subsection:riscvCompliance}, to be safe, these compliance tests along with any other unit tests designed during this thesis will then be run on the \ac{FPGA} processor.

\section{Recompiler \& Assembler Specifications}
\subsection{Input and Output Relationship}
\subsubsection{Quantitative Description}
\subsubsection{Justification}
\subsubsection{Measurement}

\subsection{RISC-V Compliance}
\subsubsection{Quantitative Description}
\subsubsection{Justification}
\subsubsection{Measurement}

\section{List of Acronyms}
% @Author: AnthonyKenny98
% @Date:   2019-10-03 19:49:49
% @Last Modified by:   AnthonyKenny98
% @Last Modified time: 2020-03-01 12:54:31
\begin{acronym}
\acro{API}{Application Programming Interface}
\acro{ARM}{Advanced RISC Machine}
\acro{CDC}{Collision Detection Circuit}
\acro{CISC}{Complex Instruction Set Computer}
\acro{CPI}{Cycles Per Instruction}
\acro{FPGA}{Field Programmable Gate Array}
\acro{CPU}{Central Processing Unit}
\acro{GPU}{Graphics Processing Unit}
\acro{GUI}{Graphical User Interface}
\acro{HDL}{Hardware Description Language}
\acro{ISA}{Instruction Set Architecture}
\acro{PRM}{Probabalistic Road Map}
\acro{RISC}{Reduced Instruction Set Computer}
\acro{RRT}{Rapidly-exploring Random Tree}
\acro{RRT*}{Rapidly-exploring Random Tree Star}
\acro{RTOS}{Real-Time Operating Systems}
\acro{RV32I}{RISC-V 32-Bit Integer}
\acro{SoC}{System on Chip}
\acro{UAV}{Unmanned Aerial Vehicle}
\acro{2D}{2-Dimensional}
\acro{3D}{3-Dimensional}
\end{acronym}

\todo[inline]{Use glossary package}
\todo[inline]{Use better acronym package that includes plurals}


\clearpage

\bibliography{techspecs}
\bibliographystyle{ieeetr}


\end{document}  


\newpage
\section{IEEE Standard for Floating-Point Arithmetic}
\label{section:honeybee_appendix_ieee}
    Integers can be represented in binary very easily. The \gls{LSB} carries the value of $2^0$, the next, $2^1$, and so on. Examples shown below for a 32-bit integer:\\

    \begin{centering}
    \begin{tabular}{c c}
    $00000000 00000000 00000000 00000001$ = & $1$ \\
    $00000000 00000000 00000000 00000100$ = & $4$ \\
    $00000000 00000000 00000000 00010101$ = & $21$ \\
    \end{tabular}
    \end{centering}

    Representing a floating-point fractional number in binary is also simple. Consider the fractional number 1.75. It can be represented in point binary as 1.11. This is converted back into decimal using the same logic as above:
    $$2^0 + 2^{-1} + 2^{-2} = 1.75 $$

    However, in computers, there is no way to represent the decimal point. All that is available is a (32-bit, for example) sequence of 1s or 0s. How does a processor know where to put the decimal point? \gls{IEEE754} is a technical standard for representing floating-point numbers and conducting floating-point arithmetic.
    A 32-bit floating-point number is represented by a 1-bit sign, an 8-bit exponent, and a 23-bit fraction, or ``mantissa''. The sign is simple; 0 for a positive number, 1 for a negative number. Now consider the fractional binary number 1111.11 (= 15.75). Similarly to regular scientific notation, this can be represented with an exponent:

    $$1111.11 = 1.11111 \times 2^3$$

    Since we are working with binary values, we use base 10. To prove this: 

    $$1.11111 \rightarrow  \Big(2^0 + 2^{-1} + 2^{-2} + 2^{-3} + 2^{-4} + 2^{-5}\Big) \times 2^3   = 1.96875 \times 2^3 = 15.75$$

    In this same way, the mantissa is multiplied by 2 to the power of the exponent and the sign is added. The mantissa has an implied leading 1, so the 23 bits shown are those to the right of the decimal point. Examples are included below: \\
    \begin{table}[H]
    \begin{centering}
    \begin{tabular}{|c|c|c|c|}
    \hline
    \textbf{sign} & \textbf{exponent} & \textbf{mantissa} & \textbf{Decimal}\\
    \hline
    0 & 01111111 & 1100000 00000000 00000000 & 1.75 \\
    \hline
    1 & 01111111 & 1100000 00000000 00000000 & -1.75 \\
    \hline
    0 & 01111111 & 0000000 00000000 00000000 & 1 \\
    \hline
    0 & 10000001 & 0110000 00000000 00000000 & 5.5 \\
    \hline
    \end{tabular}
    \mycaption{IEEE-754 Floating Point Examples}{}
    \end{centering}
    \end{table}

\newpage 
\section{Mapping HoneyBee's Output Sequence to a Grid-Map}
\label{section:honeybee_appendix_mapping}
    % @Author: AnthonyKenny98
% @Date:   2020-04-09 08:08:15
% @Last Modified by:   AnthonyKenny98
% @Last Modified time: 2020-04-09 08:33:23
For an $\epsilon \times \epsilon \times \epsilon$ grids, the grid space can be represented by an $\epsilon^3$ bit sequence.
A grid's index in the sequence is a function of its coordinates. This can be thought of as a 3 dimensional nested loop, incrementing most frequently on the $x$-axis, then the $y$-axis, then the $z$-axis. An index of 0 is the \gls{LSB} and an index of $\epsilon^3 - 1$ is the \gls{MSB}.

Mathematically, this can be thought of as the following: \\

$$f(x,y,z) = x + \epsilon y + \epsilon^2 z$$

In HoneyBee, this can computed efficienty by using shifts rather than multiplications, as long as $\epsilon$ is an even number. For $\epsilon=2$, SHAMT = 1. \\

$$f(x,y,z) = x + (y << \text{SHAMT}) + (z << \text{SHAMT} << \text{SHAMT}$$

Graphically, this can be represented as shown in Figure \ref{fig:honeybee_mapping} for $\epsilon = 2$.

\begin{figure}[H]
\begin{centering}
\includegraphics[width=0.6\linewidth]{appendices/honeybee/img/mapping.png}
\mycaption{Bit Sequence Mapping for a $2\times2\times2$ Grid Space}{}
\label{fig:honeybee_mapping}
\end{centering}
\end{figure}

\section{HoneyBee Handshake Control Protocol}
\label{section:honeybee_appendix_handshake}
    % @Author: AnthonyKenny98
% @Date:   2020-04-08 12:12:17
% @Last Modified by:   AnthonyKenny98
% @Last Modified time: 2020-04-08 12:26:20
\begin{table}[H]
\begin{centering}
\begin{tabular}{|p{0.2\linewidth}|p{0.7\linewidth}|}
\hline
\textbf{Port}   & \textbf{Description} \\
\hline
\texttt{start}  &
    This signal controls the block execution and must be asserted to logic 1 for the design to begin operation. It should be held at logic 1 until the associated output handshake \texttt{ready} is asserted. When \texttt{ready} goes high, the decision can be made on whether to keep \texttt{start} asserted and perform another transaction or set \texttt{start} to logic 0 and allow the design to halt at the end of the current transaction. If \texttt{start} is asserted low before \texttt{ready} is high, the design might not have read all input ports and might stall operation on the next input read. \\
\hline
\texttt{ready}  &
    This output signal indicates when the design is ready for new inputs. The \texttt{ready} signal is set to logic 1 when the design is ready to accept new inputs, indicating that all input reads for this transaction have been completed. If the design has no pipelined operations, new reads are not performed until the next transaction starts. This signal is used to make a decision on when to apply new values to the inputs ports and whether to start a new transaction should using the \texttt{start} input signal. If the \texttt{start} signal is not asserted high, this signal goes low when the design completes all operations in the current transaction. \\
\hline
\texttt{done}   &
    This signal indicates when the design has completed all operations in the current transaction. A logic 1 on this output indicates the design has completed all operations in this transaction. Because this is the end of the transaction, a logic 1 on this signal also indicates the data on the \texttt{return} port is valid. Not all functions have a function return argument and hence not all RTL designs have an \texttt{return} port. \\
\hline
\texttt{idle}   & 
    This signal indicates if the design is operating or idle (no operation). The idle state is indicated by logic 1 on this output port. This signal is asserted low once the design starts operating. This signal is asserted high when the design completes operation and no further operations are performed. \\
\hline
\end{tabular}
\mycaption{Description of the ports associated with the handshake protocol for HoneyBee}{. Adapted from Vivado HLS User Guide\cite{Xilinx2014}}
\end{centering}
\end{table}

\section{HoneyBee Interface Synthesis Report}
\label{section:honeybee_appendix_synthesis_report}
    % @Author: AnthonyKenny98
% @Date:   2020-04-08 15:51:15
% @Last Modified by:   AnthonyKenny98
% @Last Modified time: 2020-04-08 15:58:54
\begin{table}[H]
\begin{centering}
\begin{tabular}{|l|l|l|l|l|}
\hline
\textbf{RTL Ports}  & \textbf{Direction}    & \textbf{Bits} & \textbf{Protocol} & \textbf{C Type} \\
\hline
ap\_clk  & in & 1 & ap\_ctrl\_hs & return value \\
\hline
ap\_rst  & in & 1 & ap\_ctrl\_hs & return value \\
\hline
ap\_start  & in & 1 & ap\_ctrl\_hs & return value \\
\hline
ap\_done  & out & 1 & ap\_ctrl\_hs & return value \\
\hline
ap\_idle  & out & 1 & ap\_ctrl\_hs & return value \\
\hline
ap\_ready  & out & 1 & ap\_ctrl\_hs & return value \\
\hline
ap\_return  & out & 64 & ap\_ctrl\_hs & return value \\
\hline
edge\_p1\_x   & in & 32 & ap\_none & scalar \\
\hline
edge\_p1\_y   & in & 32 & ap\_none & scalar \\
\hline
edge\_p1\_z   & in & 32 & ap\_none & scalar \\
\hline
edge\_p2\_x   & in & 32 & ap\_none & scalar \\
\hline
edge\_p2\_y   & in & 32 & ap\_none & scalar \\
\hline
edge\_p2\_z   & in & 32 & ap\_none & scalar \\
\hline
\end{tabular}
\mycaption{HoneyBee Interface Synthesis Report}{ for $\epsilon = 4$. Results taken from Vivado HLS synthesis report.}
\end{centering}
\end{table}

\section{HoneyBee-B Variants}
\label{section:honeybee_appendix_hbb_variants}
    \subsubsection{HoneyBee-A}
    Calling each Line Intersects plane from within the honeybee() function. 
    max latency = 6.66 us and no overlap between checking each plane.
    No Pragmas used, one instance of the \texttt{Check\_Planes} module, computing collisions for $xy$, $xz$, and $yz$ oriented planes sequentially.

    \subsubsection{HoneyBee-B1}
    By tweaking the way the C Code was laid out, 3 different instances of the same \texttt{Check\_Planes} module were instantiated, executing in parallel.

    \subsubsection{HoneyBee-B2}
    Added HLS PRAGMA ``function\_instantiate'' to minimise control logic depending on inputs. Instantiate an instance of 3 slightly different modules: \texttt{Check\_Planes\_XY}, \texttt{Check\_Planes\_XZ}, and \texttt{Check\_Planes\_YZ}
    Marginally faster but significantly smaller.

    \subsubsection{HoneyBee-B3, B4, B5}
    HB-4, HB-5, and HB-6 were all attempts at using the HLS UNROLL Pragma to unroll the $\epsilon$ deep loop for checking a set of planes, to not much avail. This failed to result in much of a speedup because the loop writes to the collision variable. 

    \subsubsection{HoneyBee-C}
    Use HLS PRAGMA ``PIPELINE'' to execute the $\epsilon$ iterations in parallel.
