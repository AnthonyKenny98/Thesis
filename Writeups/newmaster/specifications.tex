% @Author: AnthonyKenny98
% @Date:   2019-10-31 10:05:27
% @Last Modified by:   AnthonyKenny98
% @Last Modified time: 2019-10-31 10:47:45


\section{Overall System Specifications}
Let the overall system be defined as 2 part: the Processor Module and the Compiler/Assembler Module. \\
The following subsections detail the technical specifications of the overall system. System specifications based on comparison with benchmarks will be compared against the same programs running on an unmodified, off-the-shelf RISC-V processor synthesized on the same \ac{FPGA}.
\subsection{Speed}
\subsubsection{Quantitative Description}
This project must deliver a system that, given a program that implements, in C, an algorithm often used in autonomous drone motion planning, executes said algorithm an order of magnitude faster than a generic RISC-V processor.

\subsubsection{Justification}
This project aims to achieve a speedup of at least one order of magnitude (10 times) when compared to benchmark performance, in the execution of pure motion planning algorithm programs.
The justification for an order of magnitude speedup comes from similar projects that accelerated motion planning algorithms, acheiving speedups from between 1 and 3 orders of magnitude. These indclude approaches using external hardware accelerators\cite{Murraya}, reprogrammable hardware and parallelization on FPGAs\cite{Murray}\cite{Atay2006}\cite{Malik2015}, and chip redesigns\cite{Murrayb}\cite{Zhi}.

\subsubsection{Measurement}
There will be two broad stages of measurement for this metric. First, in simulation, the Vivado Design Suite\cite{Vivado} will allow the execution of a given compiled program to be timed on a simulated processor that is defined in an \ac{HDL}. Secondly, in synthesis, a to-be-determined tool will allow for me to time the execution of the same program, now on a processor physically synthesized on an FPGA. 

\subsection{Power Consumption}
\subsubsection{Quantitative Description}
This project must deliver a system that has comparable power consumption as an unmodified RISC-V processor operating on the same FPGA. Comparable will be defined as within a tolerance range of 10\%.
\subsubsection{Justification}
Power is defined as energy dissipated over time.\cite{AmericanElectricianHandbook} As such, when considering the application of this system in autonomous drones, we want to minimize the amount electrical energy committed to the computation of paths. Since the primary goal of this thesis is to reduce the execution time, it can aim to keep power use comparable between the benchmark system and the new system. If power remains roughly constant, but the time taken to execute a program is reduced 10 times, we should see a proportional improvement in energy efficiency.
\subsubsection{Measurement}
The Vivado Design Suite\cite{Vivado} will allow for simulated power consumption estimates, but the important measurement will be comparing the new processor to the unmodified processor on the FPGA, running the same program. I am still to determine how exactly to measure this, but the parameters for testing are known and shown above.


\section{Recompiler \& Assembler Specifications}
The first subcomponent of the system is the Compiler \& Assembler Module. It has two specifications, Correctness and Optimality. 
\subsection{Correctness}
\subsubsection{Quantitative Description}
This project must deliver a Compiler \& Assembler Module that, given a program defined in C, can compile and assemble this program into machine code, in a manner that follows the chosen ISA correctly. 
\subsubsection{Justification}
When announcing the IBM System/360 in 1964, IBM said the following: "Instruction Set Architecture is the structure of a computer that a machine language programmer must understand to write a correct program for that machine."\cite{IBM1964} That is, it is a contract between the compiler and the hardware, so that the compiler can write machine code that will work for a given computer processor. In this way, for a given ISA (whether RISC-V, or the extened RISC-V this project will design), the Compiler \& Assembler Module must adhere to that ISA to compile instruction sets that will execute correctly on the processor.
\subsubsection{Measurement}
Testing for correct compiling under the original RISC-V ISA is relatively simple. Does the Generic Compiler (which shouldn't be altered by this project) produce correct RISC-V Assembly. Then, the Recompiler module will operate on those RISC-V assembly instructions to produce a recompiled assembly instruction set that follows the project's extended RISC-V ISA. The Assembler will then assemble this into machine code that can be directly loaded into the processor's Instruction Memory. Testing for this required extensive and complete unit tests to be written during this project.

\subsection{Optimality}
\subsubsection{Quantitative Description}
This project must deliver a Compiler \& Assembler Module that, given certain extended instructions that this project defines, recompiles all regular RISC-V assembly code that is suitable for recompilation. 
\subsubsection{Justification}
The RISC-V \ac{ISA} is an ISA that supports user-level ISA extensions and specialized variants\cite{Isa2012}, which may allow the number of instructions per program to be reduced through clever redesign of a processor and new instructions. However, to reap the full performance rewards of these extensions, a compiler must use the new instructions whenever it is able.
\subsubsection{Measurement}
This will be challenging to measure. I plan to write extensive unit tests that will determine manually the optimal recomplication from RISC-V to extended RISC-V assembly, and then compare that to the output of the recompiler module. There will also be tests to check that these substitutions are occuring in larger, more complicated programs.

\section{Processor Specifications}
The second subcomponent of the system is the Processor Unit. It has two specifications, RISC-V Compliance and Syntheizability.

\subsection{RISC-V Compliance} \label{subsection:riscvCompliance}
\subsubsection{Quantitative Description}
The project must deliver a processor that is RISC-V Compliant, meaning that it can support any correctly compiled RISC-V Assembly Code.
\subsubsection{Justification}
A processor must be such that it supports the execution of all instructions defined in the \ac{ISA} for which it was designed. So too must this project's finished processor be able to correctly support any correctly compiled RISC-V assembly code. This may range from simple programs compiled into either the original or extended RISC-V \ac{ISA}, to complete operating systems, whether for drone applications or a generic linux distribution, for example. 
\subsubsection{Measurement}
The RISC-V organisation has provided a Github Repo for testing a processor for RISC-V compliance.\cite{Compliance} Once it passes this, this processor should be able to run any program or OS compiled into RISC-V assembly.

\subsection{Synthesizable}
\subsubsection{Quantitative Description}
The project must deliver a processor defined in an \ac{HDL} that is synthesizable on an FPGA for the project to be useful for drone developers. While this project will use and test with the Diligent Zync-7000 \ac{SoC}\cite{Zync}, the design should be synthesizable on most Zync boards.
\subsubsection{Justification}
Many papers that have worked in the area of accelerating motion planning algorithms for robot applications have delivered a finished product of a processor/accelerator design implemented in an \ac{HDL} for synthesis on an FPGA.\cite{Murray}\cite{Atay2006}\cite{Malik2015} This FPGA can then be used to control the robot or share computational load with a co-processor.
\subsubsection{Measurement}
Measurement for this specification is relatively simple. Once the processor is designed in an \ac{HDL}, it can be synthesized onto an FPGA using the Vivado Design Suite, given the design is synthesizable (although this is not always simple to achieve). If it is not synthesizable, the design suite will throw and error.
Finally, while the design should be correct and RISC-V compliant by the tests performed in section \ref{subsection:riscvCompliance}, to be safe, these compliance tests along with any other unit tests designed during this thesis will then be run on the \ac{FPGA} processor.